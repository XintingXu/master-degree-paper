\section{时间隐通道验证结果}
\label{chap:linphone:result}

Linphone中时间隐通道的验证,主要测试RTP模式下主动丢包方式构造隐通道的可行性,重点在于隐蔽消息能否顺利通过丢包序号进行传输。通过在Linphone源码中附加基于主动丢包的时间隐通道,并生成Android端应用,经过实际通话测试即可证明构造方法是否有效。在参数设置方面,参照\nref{chap:zigzag:results:undetectability}及\nref{chap:hash:result:undetectability}实验结果,设定$L_{Codeword}$为8,即每256个数据包就主动丢弃一个数据包。

\insertTable{
	\begin{table}
      \centering
      \caption{测试环境信息表}
      \label{tab:6:result:environment}
          \begin{tabular*}{0.8\textwidth}{@{\extracolsep{\fill}}cc}
            \toprule
            环境对象 & 详细信息 \\
            \midrule
            手机平台 & 三星 S10 Edge,Android 9 \\
            Linphone版本 & Linphone Android 4.2.3, Linphone SDK 4.3.0 \\
            编译环境 & Ubuntu 16.04,Android SDK 29,Android NDK 20b \\
            测试网络 & WiFi 2.4G,4G网络 \\
            \bottomrule
          \end{tabular*}
    \end{table}
}

如表\nref{tab:6:result:environment},测试平台为两台Android手机,型号为三星S10 Edge,系统版本为Android 9.0。Linphone源码基于Linphone Android 4.2.3,Linphone SDK 4.3.0,编译过程采用的Android NDK版本为20b,编译采用的Android SDK版本为29。测试的网络环境包括WiFi-WiFi、Wifi-4G以及4G-4G三种模式,每次发送20字节数据,判断接收方能否接收并成功还原隐蔽消息。由于Linphone需要解决NAT穿透问题,每次测试时均执行应用冷启动,并重新登录账号。

\subsection{可用性测试}
\label{chap:linphone:result:availablity}

\insertTable{
	\begin{table}
      \centering
      \caption{Linphone下时间隐通道测试成功率}
      \label{tab:6:result:availablity}
          \begin{tabular*}{0.7\textwidth}{@{\extracolsep{\fill}}ccccc}
            \toprule
            网络环境 & 测试次数 & 成功次数 & 成功率 & 平均丢包率 \\
            \midrule
            WiFi-WiFi & 50 & 47 & 94\% & 0.2\%\\
            WiFi-4G & 50 & 42 & 84\% & 2.4\%\\
            4G-4G & 50 & 40 & 80\% & 0.9\%\\
            \bottomrule
          \end{tabular*}
    \end{table}
}

可用性测试的判断标准,是隐通道接收方能否成功接收到正确的隐蔽消息。通过多种网络场景下的网络测试,模拟不同的通话场景,当各场景下的成功率均具有较高水平,即可证明基于主动丢包的时间隐通道具有可行性。不同场景下的测试结果如表\nref{tab:6:result:availablity},在不同的场景中成功率均超过了80\%。在WiFi-WiFi场景中,Linphone通过NAT穿透,建立了基于LAN的P2P链接,因此平均丢包率较低,时间隐通道具有较高的成功率。当接入4G网络时,网络复杂度增加,端到端链接稳定性较差,导致丢包率增加,时间隐通道成功率下降。

通过Linphone中时间隐通道的传输测试,证明基于主动丢包的时间隐通道具有可行性。测试中虽未采用复杂的鲁棒性策略,但实验结果表明,在测试环境中已经具有良好的可用性。对于VoLTE视频信道来说,虽然平均丢包率高于Linphone语音信道,但VoLTE双方能够直接建立P2P链接,网络环境与Linphone在WiFi-WiFi模式下通话近似。本章测试重点研究主动丢包模式是否符合SIP+RTP要求,因此采用的鲁棒性方法较简单,从而导致成功率有一定损失。对于VoLTE视频信道来说,采用\nref{chap:zigzag:model}及\nref{chap:hash:robustness}中提出的鲁棒性方案,能够取得更高的成功率。

\subsection{传输性能}
\label{chap:linphone:result:throughput}

\insertTable{
	\begin{table}
      \centering
      \caption{基于Linphone的时间隐通道性能验证对比}
      \label{tab:6:result:compare}
          \begin{tabular*}{0.6\textwidth}{@{\extracolsep{\fill}}ccc}
            \toprule
            时间隐通道方法 & 传输速率 & 信道容量 \\
            \midrule
            Zigzag-CTC & 0.88 bps & 0.009 bpp \\
            MSV-CTC & 0.73 bps & 0.007 bpp \\
            Linphone-CTC & 0.52 bps & 0.01 bpp \\
            \bottomrule
          \end{tabular*}
    \end{table}
}

由于调制解调方案方面的区别,以及Linphone语音数据包发送速率的影响,Linphone中测试得到的时间隐通道传输速率如表\nref{tab:6:result:compare}。Linphone语音数据包的发送速率为50\ packets/s,小于VoLTE视频数据包100\ packets/s的发送速率,因此,在相同的容量下传输速率减半。表\nref{tab:6:result:compare}中,Zigzag-CTC为\nref{chap:zigzag:model}中提出的基于Zigzag映射矩阵的时间隐通道构建方法,MSV-CTC为\nref{chap:hash:designation}中提出的基于多重校验的时间隐通道构建方法,两种方法的数据均面向VoLTE视频信道。Linphone-CTC即为本章实际测试得到的数据,面向Linphone语音信道。实际测试验证,基于主动丢包的时间隐通道具备有效数据传输能力,尤其适用于秘钥等关键消息传输。