\section{时间隐通道验证结果}
\label{chap:linphone:result}

Linphone中进行时间隐通道验证,主要测试RTP环境下主动丢包方式构造隐通道的可行性,重点在于隐蔽消息能否顺利通过丢包序号进行传输。通过在Linphone源码中附加基于主动丢包的时间隐通道,并生成Android端应用,以实际传输测试证明构造方法是否有效。在传输参数方面,参照本文\nref{chap:zigzag:results:undetectability}及本文\nref{chap:hash:result:undetectability}的实验结果,设定$L_{Codeword}$为8,即每256个数据包就主动丢弃一个数据包。

\insertTable{
	  \begin{table}[htbp]
      \centering
      \caption{测试环境信息表}
      \label{tab:6:result:environment}
          \begin{tabular*}{0.8\textwidth}{@{\extracolsep{\fill}}cc}
              \toprule
              环境对象 & 详细信息 \\
              \midrule
              手机平台 & 三星 S10 Edge,Android 9.0 \\
              Linphone版本 & Linphone Android 4.2.3,Linphone SDK 4.3.0 \\
              编译环境 & Ubuntu 16.04,Android SDK 29,Android NDK 20b \\
              测试网络 & WiFi 2.4G,4G网络 \\
              \bottomrule
          \end{tabular*}
    \end{table}
}

如表\nref{tab:6:result:environment},测试平台为两台Android手机,型号为三星S10 Edge,系统版本为Android 9.0。Linphone源码基于Linphone Android 4.2.3、Linphone SDK 4.3.0,编译过程采用的Android NDK版本为20b,编译采用的Android SDK版本为29。测试的网络环境包括WiFi-WiFi、Wifi-4G以及4G-4G三种模式,每次发送20字节数据,判断接收方能否接收并成功还原隐蔽消息。由于Linphone需要解决NAT穿透问题,每次测试时均执行应用冷启动,并重新登录账号。

\subsection{可用性测试}
\label{chap:linphone:result:availablity}

\insertTable{
	  \begin{table}[htbp]
        \centering
        \caption{Linphone中时间隐通道测试成功率}
        \label{tab:6:result:availablity}
        \begin{threeparttable}
            \begin{tabular*}{0.7\textwidth}{@{\extracolsep{\fill}}ccccc}
                \toprule
                网络环境 & 测试次数 & 成功次数\tnote{1} & 成功率 & 平均丢包率 \\
                \midrule
                WiFi-WiFi & 50 & 47 & 94\% & 0.2\%\\
                WiFi-4G & 50 & 42 & 84\% & 2.4\%\\
                4G-4G & 50 & 40 & 80\% & 0.9\%\\
                \bottomrule
            \end{tabular*}
            \begin{tablenotes}
                \footnotesize
                \item[1] 成功次数定义为无误码的传输次数
            \end{tablenotes}
        \end{threeparttable}
    \end{table}
}

可用性测试的判断标准,是隐通道接收方能否完整接收到隐蔽消息。通过多种网络场景下的传输测试,模拟不同的通话场景,当各场景下均具有较高成功率,即可证明基于主动丢包的时间隐通道具有可行性。不同场景下的测试结果如表\nref{tab:6:result:availablity},在各场景中成功率均超过了{80\ \%}。在WiFi-WiFi场景中,Linphone通过NAT穿透,建立了基于LAN的P2P链接,因此平均丢包率较低,时间隐通道具有较高的成功率。当接入4G网络时,网络复杂度增加,链接稳定性较差,导致丢包率增加、时间隐通道成功率下降。

通过实际传输测试,证明了基于主动丢包的时间隐通道具有可行性。测试中虽未采用复杂的鲁棒性策略,但实验结果表明,无误码的成功率已经很高。对于VoLTE视频信道来说,虽然平均丢包率高于Linphone语音信道,但VoLTE双方能够建立直接的P2P链接,网络环境与Linphone在WiFi-WiFi模式下近似。另一方面,本章测试重点为主动丢包模式是否符合SIP+RTP网络要求,因此采用的鲁棒性方法较简单,在成功率方面有一定损失。对于VoLTE视频信道来说,采用本文\nref{chap:zigzag:model}及本文\nref{chap:hash:robustness}中提出的鲁棒性方案,能够取得更高的成功率。

\subsection{传输性能}
\label{chap:linphone:result:throughput}

\insertTable{
	  \begin{table}[htbp]
        \centering
        \caption{基于Linphone的时间隐通道性能验证对比}
        \label{tab:6:result:compare}
            \begin{tabular*}{0.6\textwidth}{@{\extracolsep{\fill}}ccc}
                \toprule
                时间隐通道方法 & 传输速率 & 信道容量 \\
                \midrule
                Zigzag-CTC & 0.88 bps & 0.009 bpp \\
                MSV-CTC & 0.49 bps & 0.005 bpp \\
                Linphone-CTC & 0.52 bps & 0.01 bpp \\
                \bottomrule
            \end{tabular*}
    \end{table}
}

由于调制解调方案的区别,以及Linphone语音数据包发送速率的影响,Linphone中测试得到的时间隐通道传输速率如表\nref{tab:6:result:compare}。Linphone语音数据包的发送速率为{50\ pkts/s},小于VoLTE视频数据包的发送速率{100\ pkts/s},因此,在相同的容量下传输速率减半。表\nref{tab:6:result:compare}中,Zigzag-CTC为本文\nref{chap:zigzag:model}中提出的基于Zigzag映射矩阵的时间隐通道构建方法,MSV-CTC为本文\nref{chap:hash:designation}中提出的基于多重校验的时间隐通道构建方法,两种方法均面向VoLTE视频信道。Linphone-CTC即为本章实际测试得到的数据,面向Linphone语音信道。实际测试验证,基于主动丢包的时间隐通道具备有效数据传输能力,尤其适用于秘钥等关键消息传输。