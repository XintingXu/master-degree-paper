\section{小结}
\label{chap:linphone:conclude}

本章主要验证基于主动丢包的时间隐通道构建方法,测试其在真实的RTP通话中是否具有可行性。借助开源的Linphone平台,构建与VoLTE相同的SIP+RTP网络场景,作为基础评估环境。通过主动丢包的方式,在Linphone中构建简单的时间隐通道,并进行数据传输测试,判断隐蔽消息能否顺利传输。

Linphone中的时间隐通道实现主要包括三个部分,分别为UI层控制接口、数据传输接口及隐通道执行组件。UI层控制接口位于聊天消息框,当存在三种固定前缀时,完成隐蔽消息设定、隐蔽消息提取以及发送状态查询功能。数据传输接口位于Linphone SDK中,负责传输时间隐通道的消息及控制命令,实现UI层接口与时间隐通道执行组件的数据交互。时间隐通道执行组件添加到oRTP协议栈中,负责数据包传输控制及监听,同时实现时间隐通道的调制与解调。

通过实际传输验证,基于主动丢包的时间隐通道构建方法,能够将隐蔽数据调制到主动丢弃的数据包序号,并且符合RTP协议的传输规则。基于主动丢包的时间隐通道,虽然不需要时钟进行传输同步,但需要面对网络噪声中丢包事件的干扰。因此在Linphone环境下的测试中,由于噪声干扰导致隐蔽消息出现部分错误。本文中提出了两种基于主动丢包的时间隐通道构建方法,在鲁棒性方面均进行了针对性设计,从而能够有效降低噪声干扰。