\section{概述}
\label{chap:linphone:overview}

Linphone是开源的VoIP应用,在数据传输模式方面与VoLTE具有相似性,并且具有良好的跨平台特性。通过结合SIP(Session Initiation Protocol)及RTP技术,Linphone支持多端音视频通话及多媒体消息。在音视频通话原理中,Linphone与VoLTE存在相似性,二者均采用SIP建立端到端会话过程,并由RTP承载音视频消息。因此,在Linphone中验证时间隐通道构建方法,便于软件开发与流程设计,并且与VoLTE环境中的测试结果基本一致。

本文提出了两种时间隐通道构建方法,均采用了主动丢包的方式实现隐蔽消息调制。因此,本章重点验证其在VoIP环境中的可行性。通过软件实现及实际传输测试,判断主动丢包能否通过VoIP机制进行传递,时间隐通道的前提能否满足。

VoLTE通话中,双方设备均接入了运营商网络,由运营商保证端到端延迟及传输可靠性。而对于Linphone来说,其面临的网络环境不统一,存在LAN(Local Area Network)-LAN、LAN-WAN(Wide Area Network)、WAN-LAN及WAN-WAN等不同场景。因此,为满足全场景通信,Linphone需要通过SIP隧道实现数据中转。当无法建立直接链路时,Linphone需要数据转发完成通话,在稳定性及可靠性方面较差。此外,如表\nref{tab:2:qci-classification},运营商网络中VoIP数据包的优先级低于VoLTE数据包,即使经运营商网络也无法保证可靠性。

因此,Linphone通话中的网络噪声强于VoLTE,并且Linphone视频信道存在一定的失败几率。Linphone环境中验证时间隐通道,优先选择语音信道进行,从而在测试效果方面与VoLTE视频信道的Excellent场景近似。

该方法的创新点如下:
\begin{itemize}
	\item 在Linphone平台中,实现了基于主动丢包的时间隐通道;
	\item 通过实际传输测试,证明了基于主动丢包的时间隐通道具有有效的数据传输能力;
	\item 隐通道模块设计隐秘,隐通道工作过程无工作痕迹,有效保护数据及设备安全。
\end{itemize}

通过测试结果,验证了基于主动丢包时间隐通道的可行性,满足时间隐通道的构建基础。