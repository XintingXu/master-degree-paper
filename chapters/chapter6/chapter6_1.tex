\section{概述}
\label{chap:linphone:overview}

本文提出了两种时间隐通道构建方法,均采用了主动丢包的方式实现隐蔽消息传输。各方法的实验测试中,对抗检测能力、鲁棒性、传输性能及构建代价均进行了评估,并且实验结果满足了时间隐通道的指标要求。因此,本章通过构建时间隐通道原型系统,对主动丢包的调制模式进行验证,测试实际网络环境中是否可行。

Linphone是开源的VoIP应用,在数据传输模式方面与VoLTE具有相似性,并且具有良好的跨平台特性。通过结合SIP(Session Initiation Protocol)及RTP技术,Linphone支持多端音视频通话及多媒体消息。音视频通话原理方面,Linphone与VoLTE存在相似性,二者均采用SIP建立端到端会话,并由RTP承载音视频消息。因此,基于Linphone构建时间隐通道原型系统,便于软件开发与流程设计,并且与VoLTE环境中的测试结果基本一致。

网络环境方面,Linphone与VoLTE存在差异。VoLTE通话中,双方设备均接入了运营商网络,由运营商保证端到端延迟及传输可靠性。而对于Linphone来说,其面临的网络环境不统一,存在LAN(Local Area Network)-LAN、LAN-WAN(Wide Area Network)、WAN-LAN及WAN-WAN等不同场景。因此,为满足全场景通信,Linphone通过SIP隧道实现数据中转。当无法建立直接链路时,Linphone通过转发实现通话,在稳定性及可靠性方面较差。此外,如表\ \nref{tab:2:qci-classification},运营商网络中VoIP数据包的优先级低于VoLTE数据包,即使经运营商网络也无法保证可靠性。

Linphone通话中的网络噪声强于VoLTE,并且Linphone视频通话存在一定的失败几率。因此,原型系统选择语音信道作为载体,从而在测试效果方面与VoLTE的Excellent场景近似。

该方法的创新点如下:
\begin{itemize}
	\item Linphone平台中,实现了基于主动丢包的时间隐通道原型系统;
	\item 通过实际传输测试,证明了主动丢包时间隐通道的数据传输能力;
	\item 隐通道用户接口隐蔽,并且工作过程无痕迹,有效保护数据及系统安全。
\end{itemize}

基于通过实验测试,验证了主动丢包时间隐通道的可行性,满足了时间隐通道的构建基础。