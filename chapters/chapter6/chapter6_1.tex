\section{概述}
\label{chap:linphone:overview}

Linphone是开源的VoIP应用,在数据传输模式方面与VoLTE具有相似性,并且具有良好的跨平台特性。通过结合SIP(Session Initiation Protocol)及RTP技术,Linphone支持多端音视频通话及多媒体消息。在音视频通话原理中,Linphone与VoLTE存在相似性,二者均采用SIP建立端到端会话过程,并将音视频消息通过RTP承载。因此,在Linphone中验证时间隐通道构建方法,能够节省软件开发与设计流程,并且得到与VoLTE环境中相似的结果。

本文中提出了两种时间隐通道构建方法,均采用了主动丢包的方式实现隐蔽消息调制。因此,本章的重点验证内容为基于主动丢包的构建方式,判断其在VoIP环境中的可行性。通过验证传输结果,判断主动丢包能否通过VoIP机制进行传递,从而满足构造时间隐通道的前提。

对于VoLTE通话场景来说,进行通话的双方设备均接入了运营商网络,在端到端延迟及传输可靠性方面有运营商进行保证。而对于Linphone来说,其面临的网络环境不具备统一性,存在LAN-LAN、LAN-WAN、WAN-LAN及WAN-WAN等不同场景。因此,为满足全场景通信,Linphone需要通过SIP隧道实现数据中转。当无法建立直接链路时,Linphone受数据转发的影响,在稳定性及可靠性方面较差。此外,如表\nref{tab:2:qci-classification},运营商网络中VoIP数据包的优先级低于VoLTE数据包。因此,在对时间隐通道的验证中,选择Linphone语音信道进行时间隐通道构建测试。

通过构建结果测试,验证了基于主动丢包的时间隐通道,在可行性方面具备数据传输能力,满足时间隐通道的构建基础。