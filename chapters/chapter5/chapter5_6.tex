\section{本章小结}
\label{chap:hash:summary}

本章主要介绍基于多重校验纠错的时间隐通道构建方法,重点降低主动丢包时间隐通道的传输误码率。为提高时间隐通道鲁棒性,该方法采用了多重校验纠错模式,包括基于HASH的码字间校验、基于CRC的码字自校验以及基于异或的映射矩阵校验。经过实验结果分析证明,该时间隐通道在保证传输隐蔽性的前提下,通过灵活组织传输参数,能够在降低误码率的同时保证一定的传输性能。Excellent场景中,误码率低于{0.08\ \%}的情况下,同时保持{0.49\ bps}的传输性能,较本文\nref{chap:zigzag:model}中提出的方法有了提升。网络噪声较强的Good场景中,该方法虽然损失了一定的性能,但依然能够在误码率低于{1\ \%}的情况下保持{0.2\ bps}的传输性能,有效证明了多重校验纠错的有效性。

在保密性方面,通过结合用户自定义信息及RTP头中的随机字段,实现散列值计算过程的加盐。同时,通过迭代伪随机数生成器,对每一组符号添加随机偏移量。从而保证即使相同的数据,在不同时刻产生的丢包位置也不完全相同。另一方面,即使监听者知晓了传输原理,由于缺乏传输参数等关键信息,反向破解复杂度较高,隐蔽消息得到保护。对于主动式监听者来说,只有将与序号相关的所有字段进行重设,才能完全阻止该时间隐通道。

因此,基于多重校验纠错的时间隐通道构建方法,在抗检测性、鲁棒性、保密性、调制代价及传输性能各方面,均满足了时间隐通道指标的要求。