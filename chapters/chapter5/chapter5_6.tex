\section{本章小结}
\label{chap:hash:summary}

本章主要介绍了基于多重校验的时间隐通道构建方法,在VoLTE视频信道中,通过主动丢包的方式构建时间隐通道,受噪声干扰明显,传输误码率较高。为提高时间隐通道鲁棒性,该方法设计了多重校验模式,包括基于HASH的码字间校验、基于CRC的码字自校验以及基于异或的映射矩阵校验。经过实验分析证明,该时间隐通道在保证传输隐蔽性的前提下,通过灵活组织传输参数,能够在降低误码率的同时保证一定的传输性能。根据实验分析,Excellent场景中,能够保证误码率低于{0.08\ \%}的情况下,保持{0.49\ bps}的传输能力,较本文\nref{chap:zigzag:model}中提出的方法有了显著提升。网络噪声较强的Good场景中,该方法虽然损失了一定的性能,但依然能够在误码率低于{1\ \%}的情况下保持{0.2\ bps}的传输性能,有效证明了多重校验方法的有效性。

在保密性方面,通过结合用户自定义信息及RTP头中的随机字段,实现摘要加盐,监听者无法还原校验内容。同时,通过迭代伪随机数生成器,对每一组符号添加随机偏移量。从而保证即使传输相同的数据,在不同的通话时刻产生的相对丢包位置也完全不同。即使监听者获取了传输原理,由于缺乏传输参数等关键信息,反向破解复杂度较高,隐蔽消息得到保护。对于主动式监听者来说,只有将所有与序号相关的字段进行重设,才能完全阻止该时间隐通道。

因此,基于多重校验的时间隐通道构建方法,在抗检测性、鲁棒性、保密性、调制代价及传输性能各方面,均满足了时间隐通道的设计要求。