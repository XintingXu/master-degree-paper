\section{概述}
\label{chap:hash:overview}

通过主动丢包的方式构建时间隐通道,为保证传输隐蔽性,主动丢包率较低。通过分析抓包数据,VoLTE视频信道丢包率高于LTE网络的QCI目标值,时间隐通道的信噪比较低。VoLTE中的丢包类型,分为长度为1的随机丢包噪声,以及连续丢包噪声。对于密度较低的随机丢包噪声,通过嵌入校验信息即可有效区分噪声及信号;而对于连续丢包噪声,单组校验信息已经无法有效鉴别,需要通过多组校验信息中的冗余数据进行鉴别。

基于多重校验的时间隐通道构建方法,主要研究在码字、符号及映射矩阵三个层次中,分别添加校验过程,逐层降低噪声强度。在该时间隐通道构建方法中,采用了码字间校验、码字自校验以及符号间校验三个校验层次。结合HASH摘要算法、CRC散列算法及异或校验算法几种方式,在Excellent场景中实现了极低误码率,有效降低了Good场景中的平均误码率。

该方法的创新点如下:
\begin{itemize}
	\item 提出了基于多重校验的时间隐通道构建方法,有效降低了误码率水平;
	\item 设计了包含码字间校验、码字自校验及符号间校验三个层次的校验方式,逐层降低噪声水平;
	\item 支持传输参数调整,在增强鲁棒性的同时保证传输性能。
\end{itemize}

通过实验测试证明,通过调整该时间隐通道的各项传输参数,能够在传输性能、误码率水平及抗检测能力方面实现平衡,在可用性及有效性方面实现了提升。