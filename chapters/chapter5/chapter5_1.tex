\section{概述}
\label{chap:hash:overview}

通过主动丢包的方式构建时间隐通道,为保证传输隐蔽性,主动丢包的密度较低。通过分析实际抓包得到的数据,VoLTE视频信道中的丢包率高于LTE网络中QCI目标值,对时间隐通道来说信噪比较低。通过分析VoLTE视频数据包的丢包类型,可以划分为长度为1的随机丢包噪声,以及连续的丢包噪声。对于密度较低的随机丢包噪声,隐通道中嵌入校验信息即可有效区分噪声及信号;而对于连续的丢包噪声,单组校验信息已经无法有效鉴别,需要将其分散到不同的分组中,通过多个校验信息中的冗余数据进行鉴别。

基于多重校验的时间隐通道构建方法,主要研究在码字、符号及映射矩阵等多个层次中,分别添加校验过程,逐层降低噪声强度,降低时间隐通道的误码率水平。在该时间隐通道构建方法中,采用了码字间校验、码字自校验、符号间校验及分组映射矩阵三个层次的校验过程,结合HASH校验、CRC校验及异或校验几种方式,在Excellent场景中实现了极低误码率,有效降低了Good场景中的平均误码率。

通过实验测试证明,通过调整该时间隐通道的各项传输参数,能够在传输性能、误码率水平及抗检测能力方面实现平衡,在可用性及有效性方面实现了提升。