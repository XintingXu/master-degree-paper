\section{VoLTE音视频传输方案}
\label{chap:backinfo:volte}

%VoLTE实现概述

\insertFigure{
	\begin{figure}[htbp]
		\centering
        \includegraphics[width=0.95\textwidth]{chapters/chapter2/figures/volte-model.pdf}
        \caption{VoLTE视频通话中的数据流}\label{fig:2:volte-model}
	\end{figure}
}

语音通信是移动通信技术的基本需求,随着智能终端的发展,移动数据需求超过了语音通信需求。为满足用户移动宽带业务需求,基于LTE的第四代移动通信技术采用了全IP网络,实现了语音主导到数据主导的转换。新的传输模式下,LTE对数据传输进行了优化,同时提升了音视频通话业务能力。

\subsection{VoLTE数据处理流程}
\label{chap:backinfo:volte:datastream}

2G及3G时代,基于电路交换的音频通话技术,是支撑语音业务的核心解决方案。进入4G时代,核心网络已经转变为全数据包交换,基于电路交换的音频通话已经无法完全兼容,需要全新的解决方案。于是,类似VoIP音视频通话,基于LTE的VoLTE成为4G时代的音视频通话解决方案。实际应用中,支持VoLTE的终端能够快速建立呼叫,并且支持回落到3G或2G网络,从而兼容多种设备及场景\nupcite{poikselka2012voice, 8315208}。

如图\ \nref{fig:2:volte-model},VoLTE视频通话中,存在三个数据信道,分别为信令信道、语音信道及视频信道,所有的信道均通过数据包传输。采用音视频分离设计,VoLTE兼容了多种通话场景。通过信令信道协商通话模式及参数,决定语音信道及视频信道采用的编码方式。相较2G及3G网络,LTE数据上行速率得到提升,能够支撑更多的终端进行高清晰度通话\nupcite{ZHANG201929}。

\insertTable{
    \begin{table}[htbp]
        \centering
        \caption{LTE业务QCI分配表概述}
        \label{tab:2:qci-classification}
        \begin{threeparttable}
            \begin{tabular*}{\textwidth}{@{\extracolsep{\fill}}cccccc}
                \toprule
                QCI\tnote{1}分类 & 承载要求 & 优先级 & 可接受延迟 & 可接受丢包率 & 服务对象 \\
                \midrule
                1 & 保证QoS\tnote{2} & 2 & 100 ms & $10^{-2}$ & VoLTE语音 \\
                2 & 保证QoS & 4 & 150 ms & $10^{-3}$ & VoLTE视频 \\
                5 & 不保证QoS & 1 & 100 ms & $10^{-6}$ & VoLTE信令 \\
                7 & 不保证QoS & 7 & 100 ms & $10^{-6}$ & 交互式音视频应用 \\
                \bottomrule
            \end{tabular*}
            \begin{tablenotes}
                \footnotesize
                \item[1] QCI指Quality of Service Class Identifiers,QoS分类标签
                \item[2] QoS指Quality of Service,服务质量
            \end{tablenotes}
        \end{threeparttable}
    \end{table}
}

相较于固网,移动无线网络的噪声干扰不可忽视,用户通话体验与噪声强度相关。如表\ \nref{tab:2:qci-classification},根据LTE业务分类,VoLTE信令数据包具有1级最高优先级,可接受时延为{100\ ms},可接受丢包率为$10^{-6}$;VoLTE语音数据包具有2级优先级,可接受时延为{100\ ms},可接受丢包率为$10^{-2}$;VoLTE视频数据包优先级降为4级\nupcite{8329138},可接受时延为{150\ ms},可接受丢包率为$10^{-3}$。另一方面,即使VoIP应用的数据包经LTE网络传输,其服务质量也低于VoLTE,运营商网络只保证尽力传输。VoIP应用的音视频数据对应的优先级为7级\nupcite{7154042,6996582,Li:2015:IVS:2810103.2813618},可接受时延为{100\ ms},可接受丢包率为$10^{-6}$。

因此,VoLTE较其它VoIP应用,受网络调度产生抖动的几率要小,传输更加稳定。因此时间隐通道的操作空间较小,对构建方法提出了挑战。

\subsection{VoLTE数据包传输特征}
\label{chap:backinfo:volte:packets}
%VoLTE音视频分流处理的模式(分清楚执行组件),结合传输的逻辑设计
支持VoLTE的终端设备进行视频通话,参与通话的处理器通常由两部分组成。其中一个是AP(Application Processor)也就是应用处理器,另一个是BP(Baseband Processor)也就是基带处理器。AP执行操作系统的通用数据处理,BP负责无线通信数据处理,二者相互补充,共同完成智能终端的数据处理任务。

\insertFigure{
	\begin{figure}[htb]
		\centering
        \includegraphics[width=0.55\textwidth]{chapters/chapter2/figures/ap-bp.pdf}
        \caption{VoLTE视频通话时AP与BP的功能划分}\label{fig:2:ap-bp}
	\end{figure}
}

如图\ \nref{fig:2:ap-bp},AP与BP处理不同的媒体类型。对于VoLTE语音数据,由基带处理器按照设定的时间间隔,完成模拟信号的采样、编码,并将打包好的语音数据包通过射频系统传输。对于VoLTE视频数据,由于图像处理模块集成在应用处理器中,视频数据需要应用处理器参与处理。应用处理器调用摄像系统驱动,获取编码后的视频数据,并以视频帧为单位进行打包,得到RTP视频数据包。视频数据包由系统内核中的协议栈发送,最终由BP通过射频系统传输。类似的,在接收阶段,VoLTE语音数据由基带处理器完成接收、解包、解码,并交由扬声器播放。VoLTE视频数据包则交付系统中的网络组件,完成解包、解码后,由通话软件显示在屏幕上\nupcite{ZHANG201929,guo2019volte}。

%抓包结果,分析传输特征,时间间隔、发送密度
\subsubsection{音视频数据包发送特征}
\label{chap:backinfo:volte:packets:send}

\insertFigure{
    \begin{figure}[htb]
        \centering
        \includegraphics[width=0.75\textwidth]{chapters/chapter2/figures/length-audio-video.pdf}
        \caption{VoLTE音视频数据包发送间隔示意}\label{fig:2:audio-video}
    \end{figure}
}

对于音频数据包,采用AMR-WB格式编码时,非静音期每{20\ ms}发送一次数据包,静音期不发送数据包\nupcite{8288828}。对于视频数据包,数据包发送间隔取决于视频刷新率及编码结果,具有不确定性\nupcite{zhang2019timestamp}。如图\ \nref{fig:2:audio-video},虽然VoLTE视频数据包的长度多在{1300\ 字节}左右,但仍有部分数据包的长度随机变化;另一方面,VoLTE语音数据包的发送间隔具有规律性,而视频数据包多为密集发送。

\subsubsection{视频数据包IPD分布特征}
\label{chap:backinfo:volte:packets:ipd}
如本文\ \nref{chap:backinfo:volte:packets:send}所述,VoLTE语音数据包的IPD存在规律,因此不利于构建时间隐通道。研究VoLTE视频数据包的传输特征,对VoLTE时间隐通道构建方法具有重要意义。

\insertFigure{
    \begin{figure}[htb]
        \centering
        \includegraphics[width=0.7\textwidth]{chapters/chapter2/figures/cdf-send-receive.pdf}
        \caption{VoLTE视频数据包IPD的累积分布函数}\label{fig:2:cdf-ipd}
    \end{figure}
}

如图\ \nref{fig:2:cdf-ipd},发送阶段IPD主要集中在{30\ ms}及{0\ ms}。由于视频刷新率为{30\ fps},每隔{33\ ms}产生一个新的视频帧,而视频帧又由多个数据包组成,导致发送阶段CDF曲线不平滑。接收阶段,受网络噪声及传输延迟的影响,IPD集中在$[0,\ 30]$区间内,CDF曲线趋于平滑。因此在IPD方面,视频数据包对时间隐通道的容忍度高于语音数据包。

\subsubsection{VoLTE视频数据包丢包特征}
\label{chap:backinfo:volte:packets:dropout}

\insertFigure{
    \begin{figure}[htbp]
        \centering
        \includegraphics[width=0.7\textwidth]{chapters/chapter2/figures/pmf-dropout.pdf}
        \caption{连续丢包数的概率质量函数}\label{fig:2:pmf-dropout}
    \end{figure}
}

对于VoLTE视频数据包,连续丢包数的概率质量函数如图\ \nref{fig:2:pmf-dropout}。该统计结果,来自抓包得到的58万条VoLTE视频数据包,包含低噪声及高噪声在内的多种测试场景。PMF(Probability Mass Function)统计结果显示,VoLTE视频信道中的丢包事件,主要由单个离散丢包事件组成,连续丢包事件占据的比例较小。

\subsection{基于主动丢包的时间隐通道构建}
\label{chap:backinfo:volte:scheme}
VoLTE出现丢包的原因,主要是终端与基站空口传输过程的干扰。复杂的无线网络环境中,发送端与接收端均存在丢包可能性。对于监听者,只有监听了隐通道发送方的终端设备,才有一定几率发现基于主动丢包的时间隐通道。然而,在隐通道威胁模型及实际应用中,监听者主要监听中间媒介,因此基于主动丢包的时间隐通道,只需要保证统计特征方面的一致性即可。

通过对VoLTE通话的理论及数据分析,在VoLTE场景中构建时间隐通道,尤其是通过主动丢包的方式构建时间隐通道,应当选择VoLTE视频信道作为宿主信道。与此同时,主动丢包的结果应当模拟离散丢包模式,并且在统计特征方面拟合已知结果。另一方面,VoLTE视频信道的丢包率,已经超过了QCI的预期丢包率,表明当前LTE网络不能完全保证服务质量,通过主动丢包构建时间隐通道可行。