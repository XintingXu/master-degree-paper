\section{时间隐通道应满足的指标}
\label{chap:backinfo:metric}

%构建一个好的时间隐通道,应当满足的指标是什么
时间隐通道作为一种隐蔽传输方法,对其最基本的要求是隐蔽性,也就是抗检测能力。与此同时,传输可靠性,也就是鲁棒性,在高噪声环境中也非常重要。为满足数据传输的需要,传输性能应当达到一定的传输速率。时间隐通道构建过程产生的代价,以及数据保密性也是评估的指标。

\subsection{抗检测能力}
\label{chap:backinfo:metric:undetectability}

%抗检测能力方面,如何通过现有检测方法
时间隐通道的抗检测能力指标,要求隐通道的传输特征与宿主特征保持一致。通用的特征包括IPD及数据包发送顺序,在VoLTE音视频通话中,重点关注接收过程的IPD分布。对于基于主动丢包的时间隐通道,增加了丢包数量,因此丢包事件的分布也是评估指标之一。

%如何在特征分布上,与已知统计结果保持一致
合理的抗检测能力评估,首先应该分析宿主信道的传输特征,并将其作为标准参照。在VoLTE场景下,需要持续观测得到样本分布。在一致性比较过程中,通过本文\nref{chap:backinfo:detect}中列举的方法,由多重维度综合进行评估。对于时间隐通道构建方法,在方案中应当具备可调的传输参数,从而应对不同的网络场景,提高自身隐蔽性。

\subsection{鲁棒性}
\label{chap:backinfo:metric:robustness}

%网络噪声不可避免,突发丢包长度也是不断变化的
网络噪声是不可避免的,信道中的主动监听者也会采取措施破坏隐通道。时间隐通道的抗噪声干扰能力,对隐通道的应用范围及可靠性有重要意义。

\insertEquation{
    \begin{equation}
    \label{equ:2:ber}
		BER\ =\ \frac{error\ bits}{sent\ bits}\quad (\%)
	\end{equation}
}

%如何完成有效传输的同时,减低误码率,并具备抵御强网络噪声的能力
鲁棒性测试,通常在不同噪声强度下进行,通过隐通道的调制与解调测试,计算传输错误的比例。公式(\nref{equ:2:ber}),是通用的BER(Bit Error Rate)计算公式,反映了时间隐通道抗噪声干扰的能力。

\subsection{传输性能}
\label{chap:backinfo:metric:throughput}

%时间隐通道自身的性能,相较存储隐通道较低
相较于存储隐通道,时间隐通道在性能方面存在明显劣势。\nupcite{mazurczyk2016youskyde,7122356}时间隐通道受限于调制对象的传输速率,存在性能上限。与此同时,时间隐通道的抗检测能力、鲁棒性和传输性能难以兼顾,牺牲性能保证可用性是常用的解决方式。

\insertEquation{
    \begin{equation}
    \label{equ:2:bps}
		Throughput\ =\ \frac{sent\ bits}{time}\quad (bps)
    \end{equation}
    \begin{equation}
    \label{equ:2:bpp}
        Capacity\ =\ \frac{sent\ bits}{sent\ packets}\quad (bpp)
    \end{equation}
}

%应当达到怎样的指标,从而满足实际应用要求
传输性能的评估指标,通常包括传输速率及信道容量。传输速率代表每秒传输的数据位数,以bps(bits per second)为单位,计算方法如公式(\nref{equ:2:bps})。构造合理的时间隐通道,传输速率应达到{1\ bps}左右。\nupcite{6296078,LIANG2018162,ZHANG201866}在时间隐通道中,信道容量代表每个数据包的嵌入位数,通常以bpp(bits per packet)为单位,计算方法如公式(\nref{equ:2:bpp})。

\subsection{构建代价}
\label{chap:backinfo:metric:cost}

%构造时间隐通道以后,对传输效果的影响
对于实时应用,用户对传输质量较为敏感。时间隐通道产生的构建代价,不应超过网络噪声产生的损失。对于VoLTE通话,用户对音视频质量的评价能够有效反映传输代价。\nupcite{8288828}

%是否造成传输质量大幅度降低
通过主动丢包的方式构建时间隐通道,产生的直接影响是丢包率升高。稳定的网络环境中,应当减少主动丢包的个数,反之亦然。丢包必然导致数据损失,语音通话中表现为语音缺失,视频通话中表现为画面卡顿或残影。当前关于图像及视频质量的客观评价方法,能够在一定程度上反映通话质量,从而判断构建过程产生的代价。

\subsection{保密性}
\label{chap:backinfo:metric:non-disclosure}

%保密性要求,中间人截获,并且知晓方案,如何保证信息不泄露
隐通道的存在,迫使监听者对信道安全进行分析,甚至采取主动防御的方式进行防范。根据监听者参与程度的差异,可以划分为主动式监听者及被动式监听者。被动式监听者主要进行传输特征的分析,进行数据审计。主动式监听者采取防范手段,对可能的隐通道进行阻止,通常采用数据覆写及数据包缓冲等操作。\nupcite{MAZURCZYK2019712,8786254}

%面对防御措施,不能破坏,主动监听者及被动监听者
更重要的,当监听者已经发现了隐通道的存在,并且通过某种途径获知了隐通道的构造原理,仍要保证隐蔽消息的安全性。除了基本的加密措施,隐通道的各阶段应具备随机性,引入CSPRNG(Cryptographically Secure Pseudo-Random Number Generator)随机数发生器,打乱操作之间的线性关联。