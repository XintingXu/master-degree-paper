\section{时间隐通道应满足的指标}
\label{chap:backinfo:metric}

%构建一个好的时间隐通道,应当满足的指标是什么
时间隐通道作为一种隐蔽传输方法,对其最基本的要求便是隐蔽性,也就是抗检测能力。与此同时,传输的可靠性,也就是鲁棒性,在高噪声环境中也非常重要。为满足数据传输的需要,在传输性能方面达到一定传输速率。时间隐通道构造过程产生的代价,及数据保密性也是评估的指标之一。

\subsection{抗检测性}
\label{chap:backinfo:metric:undetectability}

%抗检测性方面,如何通过现有检测方法
时间隐通道的抗检测性指标,要求隐通道的传输特征与宿主特征保持一致。通用的特征包括IPD及数据包发送顺序,在VoLTE音视频通信中,重点关注的是接收过程的IPD分布。对于基于主动丢包的时间隐通道来说,在宿主信道中增加了丢包数量,丢包事件的分布也是需要进行评估的指标之一。

%如何在特征分布上,与已知统计结果保持一致
合理的抗检测性评估,首先应该分析得到宿主信道的分布特征,并将其作为对比的标准参照。在VoLTE场景下,需要一段时间的持续观测,得到样本分布。在一致性比较过程中,使用\nref{chap:backinfo:detect}中列举的方法,由多重维度综合进行评估。对于时间隐通道构建方法来说,在方案中应当具备可调的传输参数,从而应对不同的网络场景,提高自身隐蔽性。

\subsection{鲁棒性}
\label{chap:backinfo:metric:robustness}

%网络噪声不可避免,突发丢包长度也是不断变化的
网络噪声是不可避免的,信道中监听者对隐通道的破坏是不可预期的,时间隐通道自身对噪声的抗干扰能力,对隐通道的使用范围及可靠性有重要意义。网络中的突发丢包长度是不可预计的,

\insertEquation{
    \begin{equation}
    \label{equ:2:ber}
		BER=\frac{error\ bits}{sent\ bits}(\%)
	\end{equation}
}

%如何完成有效传输的同时,减低误码率,并具备抵御强网络噪声的能力
鲁棒性测试,通常在不同噪声强度下进行,通过时间隐通道的调制与解调测试,判断传输错误发生的比例。公式(\nref{equ:2:ber}),是通用的BER(Bit Error Rate)计算公式,反映了时间隐通道看噪声干扰的能力。

\subsection{传输性能}
\label{chap:backinfo:metric:throughput}

%时间隐通道自身的性能,相较存储隐通道较低
相较于存储隐通道,时间隐通道在性能方面存在明显劣势。\nupcite{mazurczyk2016youskyde,7122356}时间隐通道在信道容量方面,受限于调制对象的传输频率,存在传输性能上限。时间隐通道的抗检测性、鲁棒性和传输性能难以兼顾,牺牲性能保证可用性是常用的解决方式。即使是相同场景下的时间隐通道,随着调制参数及实施方式的变化,传输性能也存在较大差异。

\insertEquation{
    \begin{equation}
    \label{equ:2:bps}
		Throughput = \frac{sent\ bits}{time}(bps)
    \end{equation}
    \begin{equation}
    \label{equ:2:bpp}
        Capacity = \frac{sent\ bits}{sent\ packets}(bpp)
    \end{equation}
}

%应当达到怎样的指标,从而满足实际应用要求
传输性能的评估指标,通常包括传输速率及信道容量。传输速率代表每秒能传输的数据位数,以bps(bits per second)为单位,计算方法如公式(\nref{equ:2:bps})。构造合理的时间隐通道,在传输性能上应该达到1 bps左右的传输速率。\nupcite{6296078,LIANG2018162,ZHANG201866}信道容量,在时间银通道中,代表的是每个数据包的平均嵌入位数,通常以bpp(bits per packet)为单位,计算方法如公式(\nref{equ:2:bpp})。

\subsection{构建代价}
\label{chap:backinfo:metric:cost}

%构造时间隐通道以后,对传输效果的影响
对于实时应用来说,用户对传输质量的体验较为敏感,时间隐通道产生的构建代价,不应超过网络噪声产生的损失。对于VoLTE通话来说,用户的主观评价是最有价值的,但用户的评价标准存在不统一的问题,评估方式无法直接用于在线评估。\nupcite{8288828}

%是否造成传输质量大幅度降低
通过主动丢包的方式构建时间隐通道,产生的直接影响是丢包率升高。丢包率的增长水平,与当前网络场景有关,稳定的网络环境中,应当减少主动丢包的个数,反之亦然。丢包必然导致数据负载出现损失,语音通话中表现为语音缺失,视频通话中表现为画面卡顿或残影。当前关于图像及视频质量的客观评价方法,能够在一定程度上反应通话质量,从而判断构建过程产生的质量损失。

\subsection{保密性}
\label{chap:backinfo:metric:non-disclosure}

%保密性要求,中间人截获,并且知晓方案,如何保证信息不泄露
隐通道的存在,迫使中间人Warden对信道安全性进行分析,甚至采取主动防御的方式进行处理。根据参与程度的差异,可以分为主动式Warden及被动式Warden。被动式Warden主要进行传输特征的分析,进行数据审计。主动式Warden采取防范手段,对可能的隐通道进行阻止,包括数据覆写及数据包缓冲等操作。

%面对防御措施,不能破坏,主动Warden及被动Warden
更重要的,当Warden已经发现了隐通道的存在,并且通过某种途径获知了隐通道的构造原理,通过隐通道传输的数据仍要保证安全。除了基本的加密措施,隐通道的各阶段应具备随机性,引入CSPRNG(Cryptographically Secure Pseudo-Random Number Generator)随机数发生器,打乱操作之间的线性关联。