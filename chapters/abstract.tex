%%==================================================
%% abstract.tex for BIT Master Thesis
%% modified by Xinting Xu
%% version: 0.1
%% last update: 2020-03-06
%%==================================================

\begin{abstract}

时间隐通道是一个经典的研究课题,在传统以太网及移动互联网环境中均有广泛的生存空间。时间隐通道能够突破系统的安全限制,对监听者透明的情况下实现隐蔽消息传输。VoLTE(Voice over LTE)是基于LTE(Long Time Evolution)的高清音视频通话方案,相较以太网及移动数据网络有特定传输特征。VoLTE场景中,现有的时间隐通道构建方法存在局限性,无法满足传输隐蔽性。为提升VoLTE中隐通道传输能力,本文研究通过主动丢包的方式,在VoLTE视频信道中构建隐蔽、鲁棒的时间隐通道。本文研究内容主要包括四个部分,涵盖时间隐通道检测方法、时间隐通道构建方法,以及主动丢包隐通道传输测试,主要工作如下:
\begin{itemize}
   \item 提出了一种VoLTE环境下,对主动丢包时间隐通道的检测方法。该方法通过判断分布一致性,对样本的IPD分布、突发丢包长度分布及区间丢包数分布进行检测。通过整合多种检验方式,判断是否存在隐通道。经过实验测试,该方法能够有效检测出主动丢包率高于0.4\ \%的时间隐通道。
   \item 提出了一种基于Zigzag映射矩阵的时间隐通道构建方法。通过Zigzag映射矩阵建立码字与符号间的非线性映射,结合基于CRC的校验方法,实现了隐蔽、鲁棒、低代价的时间隐通道。实验结果表明,该方法在传输性能达到0.88\ bps的同时,误码率水平低于1.5\ \%,并且具有较强的保密性。
   \item 提出了一种基于多重校验的时间隐通道构建方法,重点提高隐通道鲁棒性。通过结合基于HASH的码字间校验、基于CRC的码字自校验以及基于异或的映射矩阵校验,有效降低了传输误码率。实验结果表明,在传输性能达到0.49\ bps时,误码率水平不高于0.08\ \%,并且构造代价远小于网络噪声。通过引入信道中的随机字段,该隐通道具有极强的保密性。
   \item 在Linphone平台中,验证了基于主动丢包的时间隐通道。通过添加隐通道控制接口、隐通道消息接口及隐通道执行组件,实现了隐蔽消息发送与接收。经实际传输测试,基于主动丢包的时间隐通道具有有效的数据传输能力,本文提出的构建方法切实可行。
\end{itemize}
\keywords{时间隐通道; VoLTE; 主动丢包; 鲁棒性; 抗检测性}
\end{abstract}

\begin{englishabstract}

   Covert timing channel is a classic research topic, it exists in IP ethernet and advanced mobile network. The advancement of the covert timing channel is that it escapes from current security utilities like firewalls and ACL (Access-Control List). As a result, a covert message is transmitted to the receiver secretly. Based on LTE (Long Time Evolution), VoLTE (Voice over LTE) provides high-definition audio and video service. However, the statistical features of VoLTE packets are distinct from the former environment, thus existing covert timing channel schemes are not suitable for VoLTE. To address this issue, covert timing channels based on active packet dropout are proposed, which provides stealthy covert communication over VoLTE. The research of this thesis focuses on the detection of dropout-based covert timing channels, covert timing channel schemes, and reality transmission tests. The contributions are summarized as four aspects: 
\begin{itemize}
   \item A comprehensive CTC (covert timing channel) detection method is proposed, which aims at dropout-based CTC over VoLTE. The method utilizes multi statistical tools, which compare the distribution difference of inter-packet delay, burst length, and packet dropout events. The experiment shows that if a CTC actively dropouts more than 0.4 \% of total packets, the method detects it out efficiently.
   \item A CTC scheme based on the Zigzag mapping matrix is proposed and evaluated. Based on a defined Zigzag mapping matrix, codewords and symbols are bonded with a nonlinear rule. With the help of CRC (Cyclic Redundancy Check) based verification, the scheme achieves low cost, stealthy, and robustness transmission. The experiment shows that the throughput of this CTC could reach 0.88 bps, while the average bit error rate is lower than 1.5 \%. Combined with mapping matrix randomize and a random salt, covert messages are non-disclosure to others.
   \item A CTC scheme based on multi-stage verification is proposed, which aims at improving robustness. To reduce the noise impact, the scheme combines HASH-based inter-codewords verification, CRC-based codeword self-verification, and mapping matrix verification. The evaluation result shows that the average bit error rate is less than 0.08 \%, even if transmission throughput remains 0.49 bps. With the random sections in RTP (Real-time Transport Protocol) header, covert messages are protected with multi-stage nonlinear distortion.
   \item The CTC based on active packet dropout is verified with Linphone, which is an open-source VoIP application. Implemented with covert channel interface in the user interface, message delivery module for the covert channel, and covert channel module in the transmission layer, the covert channel is established. With actual transmission tests, covert messages are delivered efficiently, which proves that the CTCs based on packet dropouts are feasible.
\end{itemize}
   
\englishkeywords{Covert Timing Channel; VoLTE; Active Packet Dropout; Robustness; Undetectability}
\end{englishabstract}
