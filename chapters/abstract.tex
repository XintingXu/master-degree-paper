%%==================================================
%% abstract.tex for BIT Master Thesis
%% modified by Xinting Xu
%% version: 0.1
%% last update: 2020-03-06
%%==================================================

\begin{abstract}
%本文的主要研究工作,是在VoLTE场景下,通过主动丢包的方式,构造安全、鲁棒的时间隐通道。伴随着第4代通信技术的应用,基于数据包交换的VoLTE(Voice over LTE)技术,取代了基于电路交换的传统通话技术。得益于LTE网络的高带宽、流量细分能力,VoLTE不仅为用户提供了低延迟、高清晰度的语音通话,也将低延迟的视频通话变为可能。
%借助实时传输协议RTP(Real-time Transport Protocol),VoLTE音视频数据被切分为独立的流,利用数据包优先级的差异,在保证传输效果的同时,提高抗干扰能力。对于语音数据流来说,基带内集成的信号采集单元,直接将编码后的数据打包为RTP数据包,经LTE组件发送到基站进行传输。另一方面,视频数据的来源是智能终端的摄像头模块,需要由应用处理器完成图像信息采集、压缩、编码,并按照RTP格式进行打包,最终交付基带处理器进行发送。

%由于处理途径及优先级方面的差异,数量较少的语音数据包,能够保持稳定且低丢包率的传输效果;而数量众多的视频数据包,受空口噪声的干扰,产生丢包现象不可避免。另一方面,VoLTE的通话双方均为无线终端,两侧的噪声强度和干扰阶段是不同的,丢包事件可能发生于任意一方、任意阶段,对用户来说,无法区分产生丢包的实际位置。因此,通过丢弃具有特定数据包序号的数据包,构造时间隐通道的设计方案,符合实际应用场景的约束,具有可行性。

%通过分析VoLTE视频流及语音流特征,结合实际通话测试结果,选择VoLTE视频流作为隐蔽消息的传输载体。经过统计分析,VoLTE丢包的主要类型为离散的丢包事件,并夹杂随机长度的连续丢包事件。类比时间隐通道的分析方法,对VoLTE丢包特征的分析,由分布统计和分布一致性检验两个维度组成。通过统计视频数据包在IPD(inter-packet delay)、突发丢包长度、窗口内丢包数量的分布情况,得到对应的分布特征。对独立的分布特征进行对比,即可得到不同视频流之间的差异,通过多维度的分析检测,对时间隐通道的抗检测性能力的提升,有重要的参考价值。

%基于Zig-Zag矩阵的时间隐通道方案,将原始的隐蔽消息,嵌入到各个数据包区间的丢包位置中,并引入HASH校验算法,提高传输过程的鲁棒性。对于待发送的隐蔽消息,按照设定的参数,首先切分为固定长度的消息块。接下来,为每一个数据块生成同等长度的校验块。然后按照设定的Zig-Zag矩阵,将消息块及校验块映射到消息符号,也就是数据包序号的相对偏移量。最后,将消息符号转换为数据包序号,并在发送队列中进行剔除。通过添加额外的校验块,解调过程中降低网络噪声的干扰;经过Zig-Zag矩阵的逆映射,噪声中的连续丢包被映射到离散的消息值,结合校验块完成去噪过程;调整传输参数后,该方案能够通过常用的抗检测性方案。

%基于多重HASH校验的时间隐通道方案,通过添加多重校验块、采用自校验的映射矩阵,在保证抗检测性的基础上,进一步提高了鲁棒性。隐蔽消息按照设定的参数,切分为定长消息块,接下来基于HASH的组间校验块被拼接到消息块末尾。接下来,在每个消息块及校验块组合的基础上,拼接基于CRC的自校验块,构成码字。根据码字的字长,码字被转换为代表组内序号的符号,并为每一组符号添加随机偏移量。最后,参照映射矩阵的定义,将符号映射为相对偏移量,最终转换为数据包序号。在发送队列中,将具有目标序号的数据包丢弃,调制过程完成。通过进行抗检测性测试、鲁棒性测试及性能测试,该时间隐通道方案具有充分的网络适应能力,在各方面的表现达到隐通道的基本要求。

%修改为以下表述方式

(摘要最后总结,第六章,基于Linphone的时间隐通道验证研究,如何组织待定)

时间隐通道是一个经典的研究问题,对。。。事关重要。

VoLTE是一种新型的网络环境,采用了新的设计,原有的隐通道存在不适应的问题。

本文提出了解决方案,解决了该问题。本文针对VoLTE时间隐通道检测方法、基于Zigzag矩阵的时间隐通道构造方法、基于多重校验的时间隐通道构造方法四个方面展开研究。

\begin{itemize}
\item 提出了一种VoLTE环境下,对基于主动丢包的时间隐通道的检测方法。利用……方法,实现了对……的检测。实验结果分析表明,……是有效可行的。

\item 提出了一种基于Zigzag矩阵的时间隐通道构造方法。通过……处理方法,结合……,实现了鲁棒性好,抗检测能力强的时间隐通道构造方法。实验结果分析表明,该方法可以实现误码率低于……,传输性能不低于……。

\item 提出了一种基于多重校验的时间隐通道构造方法。通过添加组间校验、码字自校验及映射矩阵的校验,有效保证了传输过程的鲁棒性。实验结果分析表明,该方法可以实现误码率低于……,传输性能不低于……。
\end{itemize}

\keywords{时间隐通道; VoLTE; 主动丢包; 鲁棒性; 抗检测性}
\end{abstract}

\begin{englishabstract}

   In order to exploit …….
   
\englishkeywords{Covert Timing Channel; VoLTE; Packet Dropout; Robustness; Undetectability}

\end{englishabstract}
