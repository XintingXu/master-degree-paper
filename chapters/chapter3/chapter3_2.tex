\section{研究背景和动机}
\label{chap:analyze:motivation}

%当前通过主动丢包方式,构造时间隐通道的方法比较少,缺少对应的检测方法
时间隐通道检测方法,与时间隐通道构建方法是一一对应的。对于传统的时间隐通道,调制对象为IPD,有多种检测方法能够与之对应。\nupcite{Cabuk:2004:ICT:1030083.1030108,6222709,Gianvecchio:2007:DCT:1315245.1315284,ARCHIBALD2014284,7346833}这些方法的基本模式是相同的,首先统计宿主信道的IPD分布特征,作为标准参照。检测过程中,统计样本的IPD分布特征,得到CDF曲线或PMF曲线,通过评估与标准参照的差异,判断是否存在时间隐通道。

在VoLTE视频通话中,丢包事件占据的比例并不高,对IPD分布的影响有限。但发生丢包事件时,产生了可观测的突发丢包长度及窗口丢包数。网络噪声越强,丢包数越多,观测到的分布变化越明显。另一方面,由于网络抖动产生的丢包,通常集中分布在一定区间内。将数据包按照设定长度划分为独立区间,分别统计区间内的丢包数量,并将所有区间的统计结果汇总,得到区间丢包的数的统计分布。基于主动丢包的时间隐通道通常有固定的丢包周期,导致区间丢包数增加,分布间出现差异。

%研究的重点,是在VoLTE场景下,构建一套时间隐通道的检测方法
除了观测对象的差异,传统的检测方法采用的数学方法比较单一。\nupcite{5590253,Gianvecchio:2007:DCT:1315245.1315284,7427126,7827996}除了常用的分布曲线检测、基于熵的检测,分布曲线之间的相对距离,也是一种重要的判别方法。\nupcite{ramdas2015wasserstein,10.5555/2926296.2926299,bellemare2017cramer}

%并作为基础,验证后续的时间隐通道构造方法
该检测方法,评估了多种观测对象,并且结合了多维度的评估方式,提升了对主动丢包时间隐通道的检测能力。通过模拟实验表明,该方法的各子项能够相互补充,有效提高了检测精度。此外,该方法用于验证本文提出的时间隐通道构建方法,评估构建方法在抗检测能力方面是否符合指标要求。