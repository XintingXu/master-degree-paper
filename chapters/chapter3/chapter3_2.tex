\section{研究背景和动机}
\label{chap:analyze:motivation}

%当前通过主动丢包方式,构造时间隐通道的方法比较少,缺少对应的检测方法
隐通道的检测方法,是依据构建方法针对性研究提出的。传统的时间隐通道,调制过程的嵌入对象为IPD,检测方法具备通用性。\nupcite{Cabuk:2004:ICT:1030083.1030108,6222709,Gianvecchio:2007:DCT:1315245.1315284,ARCHIBALD2014284,7346833}这些方法的核心思想,是统计宿主信道的IPD分布特征,作为标准参照。检测过程中,统计样本的IPD分布特征,得到CDF曲线或PMF曲线,与标准参照进行对比。

丢包事件在VoLTE视频通话中,占据的比例并不高,对IPD分布的影响有限。仅通过IPD分布进行隐通道检测,隐通道产生的影响弱于网络噪声造成的变化,检测结果不具备有效性。发生丢包事件时,产生可观测的突发丢包数量。丢包数量越大,传输质量越差,良好的网络环境下,离散丢包为主要丢包事件。网络抖动产生的丢包,通常集中在一定区间内,区间外的传输过程丢包几率较低。数据包传输过程,按照数据包数量划分为独立区间,分别计数区间内的丢包数量,比较汇总分布于参照分布,也是有效的检测方式。

%研究的重点,是在VoLTE场景下,构建一套时间隐通道的检测方法
除了统计对象方面的差异,传统的检测方法针对的是一种类型的隐通道,使用的数学方法单一。\nupcite{5590253,Gianvecchio:2007:DCT:1315245.1315284,7427126,7827996}除了常用的分布曲线检测、基于熵的检测,分布曲线之间的相对距离,也是判别分布差异的重要方式。\nupcite{ramdas2015wasserstein,10.5555/2926296.2926299,bellemare2017cramer}

%并作为基础,验证后续的时间隐通道构造方法
该时间隐通道检测方法,对多种观测对象,采用多种评估方式进行综合测试,以检测结果作为隐通道存在依据。模拟实验表明,该方法的各部分能够相互补充,提升了检测方法的检测能力。