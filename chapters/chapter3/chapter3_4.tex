\section{基于统计分析的时间隐通道检测方法设计}
\label{chap:analyze:statistical}

本节主要讲述,如何利用统计方法,对基于主动丢包的时间隐通道进行检测

\subsection{基于累积分布函数的检测方法设计}
\label{chap:analyze:statistical:cdf}

CDF函数的定义

CDF函数代表了什么

如何通过CDF进行检测

\subsection{基于熵的检测方法设计}
\label{chap:analyze:statistical:entropy}

熵的数学意义是什么

熵检测算法

如何通过熵判定一致性

\subsection{基于一致性检验的检测方法设计}
\label{chap:analyze:statistical:test}

一致性检验的通用解释

一致性检验使用的对象

\subsubsection{基于Kolmogorov-Smirnov test的检测方法设计}
\label{chap:analyze:statistical:test:ks}

K-S检测的数学理论

K-S测试的一致性判定

\subsubsection{基于Ansari-Bradley test的检测方法设计}
\label{chap:analyze:statistical:test:an}

A-B测试的数学理论

A-B测试的一致性判定

\subsubsection{基于Welch’s t-test的检测方法设计}
\label{chap:analyze:statistical:test:t}

T-test的数学理论

T-test的一致性判定

\subsubsection{基于Mann-Whitney rank test的检测方法设计}
\label{chap:analyze:statistical:test:mw}

M-W测试的数学理论

M-W测试的一致性判定

\subsection{基于相对距离的检测方法设计}
\label{chap:analyze:statistical:distance}

相对距离的通用解释

相对距离的测试对象

\subsubsection{Wasserstein距离}
\label{chap:analyze:statistical:distance:wasserstein}

wasserstein距离的数学理论

相对距离的一致性判定

\subsubsection{能量距离}
\label{chap:analyze:statistical:distance:energy}

Energy distance is a metric that measures the distance between the distributions of random vectors. Energy distance is zero if and only if the distributions are identical, thus it characterizes equality of distributions and provides a theoretical foundation for statistical inference and analysis. Energy statistics are functions of distances between observations in metric spaces. As a statistic, energy distance can be applied to measure the difference between a sample and a hypothesized distribution or the difference between two or more samples in arbitrary, not necessarily equal dimensions. The name energy is inspired by the close analogy with Newton's gravitational potential energy. Applications include testing independence by distance covariance, goodness‐of‐fit, nonparametric tests for equality of distributions and extension of analysis of variance, generalizations of clustering algorithms, change point analysis, feature selection, and more. doi: 10.1002/wics.1375

能量距离的数学理论

相对距离的一致性判定