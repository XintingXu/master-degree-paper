\section{面向基于主动丢包的VoLTE时间隐通道检测方法设计}
\label{chap:analyze:statistical}

本节主要介绍,在该时间隐通道中采用的检测工具,及各种检测工具及检测对象的组合设计。在进行检测前,利用抓包得到的结果,构建标准参照,涵盖的对象包括IPD、突发丢包长度及区间丢包数量的累积分布函数及概率质量函数。对于测试样本,同样计算其累积分布函数及概率质量函数结果,通过各种检测算法,比较分布之间的差异,得出隐通道检测结论。

\subsection{基于累积分布函数的检测方法设计}
\label{chap:analyze:statistical:cdf}

%CDF函数的定义
CDF的全程是Cumulative Distribution Function,也就是累积分布函数。其统计意义,为$F_{X}(x)=P (X\leq x)$,即$X\leq x$的概率。在实际应用中,CDF的作用是分析分布的变化趋势,从而判断是否存在异常点。对于离散变量,$F_{X}(x)$的计算方式如公式(\nref{equ:3:cdf})。本方法的判别基础,是检测对象的统计特征,属于离散分布。

\insertEquation{
    \begin{equation}
    \label{equ:3:cdf}
		F_{X}(x)=\sum_{-\infty}^{x}f(x)
	\end{equation}
}

%CDF函数代表了什么
如图\nref{fig:3:capture:ipd},图\nref{fig:3:capture:ipd:pmf}的频率分布存在波动,而转换为CDF后的图\nref{fig:3:capture:win-cdf}能够更好地体现变化趋势。CDF曲线上升平缓的部分,意味着出现频率较小,所占比重较低;而CDF上升迅速的部分,对应的是PMF中占比较大的部分,也就是统计结果中频率较高的部分。

%如何通过CDF进行检测
CDF曲线是最基本的检测方法,本检测方法中,首先计算IPD、突发丢包长度及区间丢包的CDF分布。接下来,绘制分布曲线,比较曲线的趋势及起始特征,尤其是曲线之间的差值。CDF检测方法,无法直接得出量化评估结果,因此,在时间隐通道检测方法中,作为初步筛选及分析方式。进一步,在CDF统计结果的基础上,进行分布一致性检验,得到确定的检测结果。

\subsection{基于分布一致性的检测方法设计}
\label{chap:analyze:statistical:test}

%一致性检验的通用解释
对于两个独立的样本,判断其是否来自同一个分布,需要借助的统计中分布一致性检验。在理想情况下,观测值出现的频率,与分布函数中的概率是对应的,观测值的分布函数在一定程度上可以代替真实的分布函数。一致性检验首先假设两个样本属于同分布,然后统计计算观测值与合理值的取值,比较判断假设是否成立。

%一致性检验使用的对象
用于一致性检验的样本数据越多,采样结果与真实分布更接近,检验结果具有更高的可信度。在\nref{chap:analyze:results}中,介绍了几种观测对象,其中IPD的分布存在明显的规律性,且不同网络环境下的分布基本一致。因此,本检测方法主要用于检测IPD分布的一致性。

\subsubsection{基于Kolmogorov-Smirnov test的检测方法设计}
\label{chap:analyze:statistical:test:ks}

%K-S检验的数学理论
Kolmogorov-Smirnov检验是常用的无参数、双样本检验方法,检验原理基于累积分布函数。\nupcite{doi:10.1080/01621459.1951.10500769,doi:10.1080/01621459.1967.10482916}如公式\nref{equ:2:ks},K-S检验基于CDF分布中的最大差异$D_{KS}$。当置信区间设定为$95\%$时,使用公式(\nref{equ:2:ks-p})计算$D_{KS,0.05}$,当$D_{KS}<D_{KS,0.05}$时,可以认为两个样本的分布式一致的。

%K-S检验的一致性判定
在实际应用中,通常根据$D_{KS,0.05}$的分布计算$p$值,当$p\leq 0.05$时,假设成立,样本来自相同的分布。
K-S检验在大样本环境中具有较高的准确度,样本数量较少时,统计的分布存在误差,最终检验结果不具有可靠性。同样需要注意的是,当样本数据增加时,$D_{KS,0.05}$会相应减小,$D_{KS}$上限降低,当采样不均匀时,出现误判结果。
因此,本检测方法中,K-S检验用于时间隐通道构建前后,相同样本的检测。

\subsubsection{基于Welch’s t检验的检测方法设计}
\label{chap:analyze:statistical:test:t}

%T-test的数学理论
Welch’s t检验,应用于两个独立的样本,主要评估样本的平均值是否存在差异。通常应用于验证两个独立的采样结果,在方差相同的情况下,是否具有相同的平均值。当样本基本符合正态分布时,该检验方法可以判断样本分布的一致性。Welch's t检验的优势,是可以应用与样本规模不一致的场景,具有较好的鲁棒性。

\insertEquation{
    \begin{equation}
    \label{equ:3:t-value}
		t=\frac{\overline{X}_{1}-\overline{X}_{2}}{\sqrt{\frac{s_{1}^{2}}{N_{1}}+\frac{s_{2}^{2}}{N_{2}}}}
	  \end{equation}

    \begin{equation}
    \label{equ:3:v-value}
		\nu \approx \frac{({\frac{s_{1}^{2}}{N_{1}}+\frac{s_{2}^{2}}{N_{2}})^{2}}}{\frac{s_{1}^{4}}{N_{1}^{2}\nu_{1}}+\frac{s_{2}^{4}}{N_{2}^{2}\nu_{2}}}
	  \end{equation}
}

%T-test的一致性判定
如公式(\nref{equ:3:t-value}),t检验首先需要计算统计值$t$,其中$\overline{X}$表示样本的均值,$s$表示样本的标准差,$N$表示样本的规模。接下来使用Welch–Satterthwaite公式计算自由度$\nu$,如公式(\nref{equ:3:v-value})。最后,根据Student's t-分布,计算在自由度$\nu$下,统计值$t$对应的概率,即可判断样本的相似性。

类似K-S检验,Welch's t检验得到的$p$值,反映了样本的一致性水平,一般取0.05作为分界值,对应两个样本分布一致的概率为95\%。因为Welch’s t检验适用范围的限制,只有在正态分布的情况下具有较好的效果,而在\nref{chap:analyze:results}中分析可知,IPD等统计特征的分布,多为偏态分布,因此,Welch's t检验作为IPD的检验方式之一。

\subsubsection{基于Mann-Whitney rank检验的检测方法设计}
\label{chap:analyze:statistical:test:mw}

%M-W检验的数学理论
Mann–Whitney rank检验,又称为Mann–Whitney U检验或Mann–Whitney–Wilcoxon检验,应用双样本无参数检验,对应的目标分布为均匀分布。不同于Welch's t检验,Mann–Whitney rank检验的目标分布为非正态分布,适用于大数据集的检验场景。

\insertEquation{
    \begin{equation}
    \label{equ:3:u-value}
		U_{i}=R_{i}-\frac{n_{i}(n_{i}+1)}{2}
	\end{equation}
}

%M-W检验的一致性判定
Mann–Whitney rank检验过程中,首先要将样本混合,然后按照顺序,为每个样本值分配排名。接下来,分别计算两个样本中的排名之和$R_{1}$及$R_{2}$。根据公式(\nref{equ:3:u-value}),分别计算两个样本的$U_{1}$及$U_{2}$,并取$U=min(U_{1},U_{2})$与$U_{0.05}$进行比较,当$U>U_{0.05}$时,即可认定样本属于同分布。

Mann–Whitney rank检验作为Welch's t检验的补充,应用于不满足正态分布的检验场景,关注的重点,是样本均值的偏离程度。因此,Mann–Whitney rank与Welch's t检验组合,作为IPD的一致性检验方法。

\insertTable{
	\begin{table}[]
        \centering
        \caption{分布一致性检测方法的适用范围}
        \label{tab:3:test-range}
        \begin{threeparttable}
            \begin{tabular*}{0.99\textwidth}{@{\extracolsep{\fill}}ccc}
              \toprule
              检验方法 & 测试对象 & 通过条件\\ 
              \midrule
              K-S检验 & IPD分布 & 隐通道构造前后,p值满足$p>0.05$ \\ 
              Welch's t检验 & IPD分布 & 同一场景下的样本,p值满足$p>0.05$ \\ 
              Mann–Whitney rank检验 & IPD分布 & 同一场景下的样本,p值满足$p>0.05$ \\ 
              \bottomrule
            \end{tabular*}
            \begin{tablenotes}
              \footnotesize
              \item[] Welch's t检验与Mann–Whitney rank检验通过一种即可
          \end{tablenotes}
        \end{threeparttable}
    \end{table}
}

如表\nref{tab:3:test-range},根据一致性检验方法的特征,及VoLTE中基于主动丢包的时间隐通道检测需求,不同的检验方法的测试通过条件存在差异。K-S检验作为独立的检验方式,验证的是IPD分布在时间隐通道构建前后的一致性。Welch's t检验及Mann–Whitney rank检验,作为互相补充的组合,共同检验同一场景下,样本与一致分布的差异,并且两种检验方法通过一种即可认定一致。

\subsection{基于熵的检测方法设计}
\label{chap:analyze:statistical:entropy}

%熵的数学意义是什么
基于熵的检测方法,主要依赖条件熵中的Kullback-Leibler散度。\nupcite{5590253,Gianvecchio:2007:DCT:1315245.1315284}K-L散度的计算方法,如公式(\nref{equ:2:kld}),$G(x)$作为参照分布,$F(x)$作为样本分布,二者在关系上具有约束。同时,K-L散度计算中,要求所有的输入概率值非0。

%如何通过熵判定一致性
基于熵的检测方法,优势在于对检测样本的实际分布不敏感,可以应用于任意类型的测试场景中。根据系统的熵增原理,不同的分布之间一定存在熵值差异,通常约定K-L散度值0.1作为分界线,当K-L散度超过0.1时,可以认为分布之间存在明显差异,不属于同分布。

\insertTable{
	\begin{table}[]
      \centering
      \caption{基于熵的检测方法适用范围}
      \label{tab:3:entropy-range}
          \begin{tabular*}{0.7\textwidth}{@{\extracolsep{\fill}}ccc}
            \toprule
            检验方法 & 测试对象 & 通过条件 \\ 
            \midrule
            \multirow{3}{*}{K-L散度} & IPD分布 & \multirow{3}{*}{K-L散度$<0.1$} \\ 
            & 突发丢包长度分布 \\
            & 区间丢包分布 \\
            \bottomrule
          \end{tabular*}
    \end{table}
}

如表\nref{tab:3:entropy-range},K-L散度检测可以应用于所有的观测特征,并且判别标准都是相同的。在实际的测试中,K-L散度结果的数量级与样本分布密切相关,因此具有非常好的灵敏度。

\subsection{基于相对距离的检测方法设计}
\label{chap:analyze:statistical:distance}

%相对距离的通用解释
相对距离体现了分布变化所需要的代价,传统意义上的距离表示的是空间上的度量,统计学中借鉴这一思想,用相对距离来度量样本分布改变为参照分布所需要的成本。
%相对距离的检验对象
相对距离的检测对象,适用于具有参照样本的双样本检测,重点关注样本与参照之间的差异程度。

\subsubsection{Wasserstein距离}
\label{chap:analyze:statistical:distance:wasserstein}

%wasserstein距离的数学理论
Wasserstein距离又称为空间传输距离,作为无参数检测方法,其评估对象是一个分布转换为另一个分布所需要的的代价。\nupcite{ramdas2015wasserstein}参照运输过程中的最小能量消耗,计算过程中需要同时考虑移动的分量及距离,最终计算总代价,反映了两个分布之间的相对距离。

\insertEquation{
    \begin{equation}
    \label{equ:3:wasserstein-distance}
		D_{w}(F(x),G(x)) = \sum{\frac{|F(x)-G(x)|}{n}}
	\end{equation}
}

%相对距离的一致性判定
在通常的计算方法中,Wasserstein距离中的每个样本点都有其权重,本检测方法中测试的是分布曲线之间的相对距离,因此设定曲线中的每个样本点都具有相同的权重。计算方式如公式(\nref{equ:3:wasserstein-distance}),其中,$F(x)$及$G(x)$对应CDF统计结果。考虑到不同场景中,$n$的取值存在差异,为统一评估标准,Wasserstein距离评估的结果为$D_{w}(F(x),G(x))\times n$。

\subsubsection{能量距离}
\label{chap:analyze:statistical:distance:energy}

能量距离是一种评估随机向量之间距离的指标,只有当两个分布完全一致时,能量距离的取值才取0。作为一种统计工具,能量距离可以用于评估样本及假设分布的距离,也可以用于评估两个独立样本之间的距离。在独立性测试、匹配度测试、无参数测试及分类划分领域,具有普遍应用能力。\nupcite{10.1002/wics.1375}

%能量距离的数学理论
\insertEquation{
    \begin{equation}
    \label{equ:3:energy-distance}
		D_{e}(F(x),G(x))=\sqrt{2\sum{(F(x)-G(x))}^{2}}
	\end{equation}
}

能量距离的计算方式如公式(\nref{equ:3:energy-distance}),在该检测方法中,评估的是分布的相似性,因此各点权值相等,$D_{e}(F(x),G(x))$代表了CDF曲线之间距离的量化评估结果。

%相对距离的一致性判定
\insertTable{
	\begin{table}[]
      \centering
      \caption{基于相对距离的检测方法适用范围}
      \label{tab:3:distance-range}
          \begin{tabular*}{0.8\textwidth}{@{\extracolsep{\fill}}ccc}
            \toprule
            检验方法 & 测试对象 & 通过条件 \\ 
            \midrule
            \multirow{3}{*}{Wasserstein距离} & IPD分布 & \multirow{3}{*}{$D_{w}(F(x),G(x))\times n<1.5$} \\ 
            & 突发丢包长度分布 \\
            & 区间丢包分布 \\
            \\
            \multirow{3}{*}{能量距离} & IPD分布 & \multirow{3}{*}{$D_{e}(F(x),G(x))<1.5$} \\ 
            & 突发丢包长度分布 \\
            & 区间丢包分布 \\
            \bottomrule
          \end{tabular*}
    \end{table}
}

基于相对距离的检测方法,适用的测试对象及通过条件,如表\nref{tab:3:distance-range}。通常$D_{w}(F(x),G(x))\times n$及$D_{e}(F(x),G(x))$与一致性概率无关,在本检测方法中,设定上限为$1.5$,如果距离计算结果超过1.5,则认定分布不一致。

\subsection{检测方法总结}
\label{chap:analyze:statistical:sum}
该时间隐通道检测方法,综合了集中不同检测原理的判别方式,并对观测到的多种特征分布进行检验。相较单一的检测方式,对基于主动丢包的时间隐通道具有更好的检测能力。检验方法与测试对象的对应关系,如表\nref{tab:3:detect-sum}所示,共14条测试子项,当通过其中的11条量化测试方法时,可以认定不存在时间隐通道;CDF检验结果存在主观差异,作为最终结论的参照。

\insertTable{
	\begin{table}[]
      \centering
      \caption{VoLTE时间隐通道检测方法汇总}
      \label{tab:3:detect-sum}
          \begin{tabular*}{0.98\textwidth}{@{\extracolsep{\fill}}cccc}
            \toprule
            编号 & 测试对象 & 检验方法 & 通过条件 \\ 
            \midrule
            1 & \multirow{6}{*}{IPD分布} & CDF检验 & 趋势基本一致 \\ 
            2 & & K-S检验 & $p>0.05$ \\
            3 & & Welch's t检验,Mann–Whitney rank检验 & 存在$p>0.05$\\
            4 & & K-L散度 & $KL-d<0.1$ \\
            5 & & Wasserstein距离 & $d<1.5$ \\
            6 & & 能量距离 & $d<1.5$ \\
            \\
            7 & \multirow{3}{*}{突发丢包长度分布} & CDF检验 & 趋势基本一致\\ 
            8 & & K-L散度 & $KL-d<0.1$ \\
            9 & & Wasserstein距离 & $d<1.5$ \\
            10 & & 能量距离 & $d<1.5$ \\
            \\
            11 & \multirow{3}{*}{区间丢包数分布} & CDF检验 & 趋势基本一致\\ 
            12 & & K-L散度 & $KL-d<0.1$ \\
            13 & & Wasserstein距离 & $d<1.5$ \\
            14 & & 能量距离 & $d<1.5$ \\
            \bottomrule
          \end{tabular*}
    \end{table}
}
