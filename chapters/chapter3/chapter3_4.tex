\section{主动丢包时间隐通道检测方法原理}
\label{chap:analyze:statistical}

本节主要介绍该检测方法采用的检测工具,及检测工具与检测对象的组合关系。在进行检测前,基于抓包结果构建标准参照,涵盖的对象包括IPD、连续丢包数及区间丢包数量,统计函数包括累积分布函数及概率质量函数。对于测试样本,首先计算其累积分布函数及概率质量函数结果,然后通过检测工具比较分布之间的差异,最终得出隐通道检测结果。

\subsection{基于累积分布函数的检测}
\label{chap:analyze:statistical:cdf}

%CDF函数的定义
CDF的全称是Cumulative Distribution Function,也就是累积分布函数。其统计意义,对应$F_{X}(x)=P (X\leq x)$,即$X\leq x$的概率。实际应用中,通过CDF分析统计分布的变化趋势,从而判断是否存在异常点。对于离散变量,$F_{X}(x)$的计算方式如公式(\nref{equ:3:cdf})。
\insertEquation{
    \begin{equation}
    \label{equ:3:cdf}
		F_{X}(x)\ =\ \sum_{-\infty}^{x}\ f(x)
	\end{equation}
}
%CDF函数代表了什么
如图\nref{fig:3:capture:ipd},即使图\nref{fig:3:capture:ipd:pmf}的频率分布存在波动,图\nref{fig:3:capture:ipd:cdf}的CDF曲线也能体现变化趋势。CDF曲线中上升平缓的部分,意味着对应频率较小,所占比重较低;而CDF曲线中上升迅速的部分,对应PMF中占比较大的部分,统计结果中的频率较高。

%如何通过CDF进行检测
CDF曲线是最基本的检测方法,本方法中,首先对比CDF曲线的差异。对于CDF曲线,重点关注曲线的趋势及起始特征,其次是曲线之间的差距。但是,基于CDF曲线的检测方法,无法直接得出量化评估结论。因此该方法中,CDF曲线仅用于初步检测。接下来,通过分布一致性检测及相对距离检测,即可得到量化评估结果。

\subsection{基于分布一致性的检测}
\label{chap:analyze:statistical:test}

%一致性检验的通用解释
判断两个独立样本是否符合相同分布,需要借助分布一致性检验方法。理想情况下,观测结果的频率与概率是对应的。因此,样本足够的情况下,观测值的概率由频率代替。一致性检验首先假设两个样本属于同分布,然后统计分布之间的差距,最终根据计算结果判断假设是否成立。

%一致性检验的适用对象
本文\nref{chap:analyze:results}中,统计对象包括IPD、连续丢包数及区间丢包数。其中只有IPD为直接观测值,并且IPD分布在不同场景中具有相同规律。因此,分布一致性检测应用于IPD分布的一致性检验。

\subsubsection{基于Kolmogorov-Smirnov检验的检测}
\label{chap:analyze:statistical:test:ks}

%K-S检验的数学理论
Kolmogorov-Smirnov检验是常用的无参数、双样本检验方法,其检验原理基于累积分布函数。\nupcite{doi:10.1080/01621459.1951.10500769,doi:10.1080/01621459.1967.10482916}如公式\nref{equ:2:ks},K-S检验基于CDF分布的最大差异$D_{KS}$。当置信区间设定为$95\%$时,通过公式(\nref{equ:2:ks-p})计算$D_{KS,\ 0.05}$,当$D_{KS}\ <\ D_{KS,\ 0.05}$时,认为两个样本的分布具有一致性。

%K-S检验的一致性判定
在实际应用中,主要根据$D_{KS,\ 0.05}$计算$p$值,当$p\leq 0.05$时判定样本分布一致。K-S检验对大样本数据具有较高的准确度,当样本数量较少时,由于统计分布存在误差,导致检验结果不完全可靠。同样需要注意的是,当样本规模增加时,$D_{KS,\ 0.05}$会相应减小,导致$D_{KS}$上限降低,如果采样不均匀则出现误判。

\subsubsection{基于Welch's t检验的检测}
\label{chap:analyze:statistical:test:t}

%T-test的数学理论
Welch's t检验,适用于双样本检验。尤其当样本基本符合正态分布时,该检验方法可以准确判断样本的分布一致性。Welch's t检验的另一个优势,是在样本规模不一致的测试中仍然有效,降低了对测试样本的约束条件。
\insertEquation{
    \begin{equation}
    \label{equ:3:t-value}
        t\ =\ \frac{\overline{X}_{1}\ -\ \overline{X}_{2}}{\sqrt{\frac{s_{1}^{2}}{N_{1}}\ +\ \frac{s_{2}^{2}}{N_{2}}}}
    \end{equation}
    \begin{equation}
    \label{equ:3:v-value}
        \nu\ \approx\ \frac{({\frac{s_{1}^{2}}{N_{1}}\ +\ \frac{s_{2}^{2}}{N_{2}})^{2}}}{\frac{s_{1}^{4}}{N_{1}^{2}\nu_{1}}\ +\ \frac{s_{2}^{4}}{N_{2}^{2}\nu_{2}}}
    \end{equation}
}
%T-test的一致性判定
如公式(\nref{equ:3:t-value}),Welch's t检验首先计算统计值$t$,其中$\overline{X}$表示样本的均值,$s$表示样本的标准差,$N$表示样本的规模。接下来通过Welch–Satterthwaite公式计算自由度$\nu$,如公式(\nref{equ:3:v-value})。最后根据Welch's t值分布,计算在自由度$\nu$下,统计值$t$对应的概率,得到一致性检验的结论。

类似K-S检验,Welch's t检验得到的$p$值,反映了样本的一致性水平。通常设定$p\ =\ 0.05$为分界值,对应两个样本分布一致的概率为95\ \%。由于Welch's t检验适用符合正态分布的样本,参照本文\nref{chap:analyze:results}中的分析结果,Welch's t检验适用于IPD分布的一致性检验。

\subsubsection{基于Mann-Whitney rank检验的检测}
\label{chap:analyze:statistical:test:mw}

%M-W检验的数学理论
Mann–Whitney rank检验,又称为Mann–Whitney U检验或Mann–Whitney–Wilcoxon检验,属于双样本无参数检验,适用的分布类型为均匀分布。由于Mann–Whitney rank检验适用的分布类型与Welch's t检验不同,二者相互补充能够提升对样本的综合检测能力。
\insertEquation{
    \begin{equation}
    \label{equ:3:u-value}
		U_{i}\ =\ R_{i}\ -\ \frac{n_{i}(n_{i}\ +\ 1)}{2}
	\end{equation}
}
%M-W检验的一致性判定
Mann–Whitney rank检验过程中,首先将样本混合,然后按照顺序为观测值分配排名。接下来,分别计算两个样本的排名和$R_{1}$及$R_{2}$。然后根据公式(\nref{equ:3:u-value}),分别计算两个样本的$U_{1}$及$U_{2}$,并取$U\ =\ min\ (U_{1},\ U_{2})$与$U_{0.05}$进行比较,当$U\ >\ U_{0.05}$时,即可认定样本属于同分布。

Mann–Whitney rank检验作为Welch's t检验的补充,应用于非正态分布场景。Mann–Whitney rank检验的重点,是样本均值的偏离程度。因此,Mann–Whitney rank检验与Welch's t检验形成组合,作为IPD分布的一致性检验方法。

\insertTable{
	\begin{table}[htbp]
        \centering
        \caption{分布一致性检测方法的适用范围}
        \label{tab:3:test-range}
        \begin{threeparttable}
            \begin{tabular*}{0.85\textwidth}{@{\extracolsep{\fill}}ccc}
                \toprule
                检验方法 & 测试对象 & 通过条件\\ 
                \midrule
                K-S检验 & IPD分布 & p值满足$p\ >\ 0.05$ \\ 
                Welch's t检验\tnote{1} & IPD分布 & p值满足$p\ >\ 0.05$ \\ 
                Mann–Whitney rank检验\tnote{1} & IPD分布 & p值满足$p\ >\ 0.05$ \\ 
                \bottomrule
            \end{tabular*}
            \begin{tablenotes}
                \footnotesize
                \item[1] Welch's t检验与Mann–Whitney rank检验通过一种即可
            \end{tablenotes}
        \end{threeparttable}
    \end{table}
}

如表\nref{tab:3:test-range},根据一致性检验方法的特征,结合VoLTE中主动丢包时间隐通道的统计特征,设定检测工具与检测对象的对应关系。K-S检验作为独立的检验方式,验证的是IPD分布在时间隐通道构建前后的一致性。Welch's t检验及Mann–Whitney rank检验,作为互相补充的组合,共同检验IPD分布在时间隐通道构建前后的一致性,并且两种检验方法通过一种即可认定一致。

\subsection{基于熵的检测}
\label{chap:analyze:statistical:entropy}

%熵的数学意义是什么
基于熵的检测基于条件熵,也就是Kullback-Leibler散度。\nupcite{5590253,Gianvecchio:2007:DCT:1315245.1315284}K-L散度的计算方法如公式(\nref{equ:2:kld}),其中$G(x)$为参照分布,$F(x)$为样本分布。K-L散度检验过程,基于样本对象的概率质量函数,并约束$f(x)$及$g(x)$均非0。

%如何通过熵判定一致性
基于熵的检测方法,优势在于对检测样本的分布不敏感,适用范围较广。根据系统的熵增原理,不同分布之间一定存在熵值差异,根据计算得到的熵增情况能够量化评估分布之间的差异。通常约定K-L散度的参照值为0.1,当K-L散度超过0.1时,认定分布之间存在差异,不属于同分布。

\insertTable{
	\begin{table}[htbp]
    \centering
    \caption{基于熵的检测方法适用范围}
    \label{tab:3:entropy-range}
        \begin{tabular*}{0.7\textwidth}{@{\extracolsep{\fill}}ccc}
            \toprule
            检验方法 & 测试对象 & 通过条件 \\ 
            \midrule
            \multirow{2}{*}{K-L散度} & IPD分布 & \multirow{2}{*}{K-L散度$<0.1$} \\ 
            & 连续丢包数分布 \\
            \bottomrule
        \end{tabular*}
    \end{table}
}

如表\nref{tab:3:entropy-range},K-L散度用于检测IPD分布及连续丢包数分布。本文\nref{chap:analyze:results:ipd}及本文\nref{chap:analyze:results:burst}的分析中,IPD及连续丢包数的概率质量函数连续,满足K-L散度检测的基本要求。而区间丢包数的概率质量函数不稳定,存在不连续的情况,因此K-L散度检测不适用于对区间丢包数的检验。

\subsection{基于相对距离的检测}
\label{chap:analyze:statistical:distance}

%相对距离的通用解释
相对距离反映了改变分布所需要的代价,相对距离越大则分布差异越大。传统意义上的距离表示的是空间度量,统计学中借鉴这一思想,采用相对距离来度量样本改变需要付出的代价。
%相对距离的检验对象
相对距离检测适用于双样本检测,检测目标为样本与参照之间的差异程度。

\subsubsection{基于Wasserstein距离的检测}
\label{chap:analyze:statistical:distance:wasserstein}

%wasserstein距离的数学理论
Wasserstein距离又称为空间传输距离,作为无参数检测方法,其结果反映了分布转换的迁移代价。\nupcite{ramdas2015wasserstein}参照运输过程中的最小能量消耗,Wasserstein距离同时考虑了移动分量及移动距离。
\insertEquation{
    \begin{equation}
    \label{equ:3:wasserstein-distance}
		D_{w}(F(x),\ G(x))\ =\ \sum{\ \frac{\left |F(x)\ -\ G(x)\right |}{n}}
	\end{equation}
}
%相对距离的一致性判定
在通用的距离计算中,每个样本点都有其权重值。然而本检测方法面向CDF曲线,因此每个样本点都具有相同的权重。CDF曲线的Wasserstein距离计算方式如公式(\nref{equ:3:wasserstein-distance}),其中$F(x)$及$G(x)$对应CDF累积值。由于不同场景中$n$的规模存在差异,为统一评估标准,实际对比的距离值为$D_{w}(F(x),\ G(x))\times n$。

\subsubsection{基于能量距离的检测}
\label{chap:analyze:statistical:distance:energy}

能量距离是评估随机向量之间距离的指标,只有当两个分布完全一致时,能量距离才为0。能量距离用于评估独立样本之间的差距,在独立性测试、匹配度测试、无参数测试及分类划分领域,具有普遍应用能力。\nupcite{10.1002/wics.1375}
%能量距离的数学理论
\insertEquation{
    \begin{equation}
    \label{equ:3:energy-distance}
		D_{e}(F(x),\ G(x))\ =\ \sqrt{2\sum{(F(x)\ -\ G(x))}^{2}}
	\end{equation}
}
能量距离的计算方式如公式(\nref{equ:3:energy-distance}),其中$F(x)$及$G(x)$对应CDF累积值,$D_{e}(F(x),G(x))$代表了能量距离计算结果。

%相对距离的一致性判定
\insertTable{
	\begin{table}[htbp]
    \centering
    \caption{基于相对距离的检测方法适用范围}
    \label{tab:3:distance-range}
        \begin{tabular*}{0.8\textwidth}{@{\extracolsep{\fill}}ccc}
            \toprule
            检验方法 & 测试对象 & 通过条件 \\ 
            \midrule
            \multirow{3}{*}{Wasserstein距离} & IPD分布 & \multirow{3}{*}{$D_{w}(F(x),G(x))\times n\ <\ 1.5$} \\ 
            & 连续丢包数分布 \\
            & 区间丢包数分布 \\
            \\
            \multirow{3}{*}{能量距离} & IPD分布 & \multirow{3}{*}{$D_{e}(F(x),G(x))\ <\ 1.5$} \\ 
            & 连续丢包数分布 \\
            & 区间丢包数分布 \\
            \bottomrule
        \end{tabular*}
    \end{table}
}

相对距离的检测结果,如表\nref{tab:3:distance-range}。通常情况下,$D_{e}(F(x),\ G(x))$及$D_{w}(F(x),\ G(x))\times n$与一致性概率无关,因此在本检测方法中,设定相对距离上限为$1.5$。如果计算结果超过$1.5$,则认定分布不一致。

\subsection{总结}
\label{chap:analyze:statistical:sum}
该时间隐通道检测方法,综合了不同原理的判别方式,并对多种分布特征同时进行检验。较单一的检测方法,该方法对基于主动丢包时间隐通道具有更好的检测能力。本方法中检验工具与测试对象的对应关系,如表\nref{tab:3:detect-sum}所示,共13条测试子项。当通过其中的10条量化测试项时,认定不存在时间隐通道。由于CDF检测结果存在主观差异,其结果作为判定参照。

\insertTable{
	\begin{table}[htbp]
    \centering
    \caption{VoLTE主动丢包时间隐通道的检测方法汇总}
    \label{tab:3:detect-sum}
        \begin{threeparttable}
            \begin{tabular*}{0.98\textwidth}{@{\extracolsep{\fill}}ccccc}
                \toprule
                编号 & 测试对象 & 检验方法 & 通过条件 & 准确性\\ 
                \midrule
                1 & \multirow{6}{*}{IPD分布} & CDF检验 & 趋势基本一致 & 低 \\
                2 & & K-S检验 & $p\ >\ 0.05$ & 中 \\
                3 & & \makecell*[c]{Welch's t检验, \\ Mann–Whitney rank检验\tnote{1}} & 存在$p\ >\ 0.05$ & 中 \\
                4 & & K-L散度 & $K-L散度\ <\ 0.1$ & 中 \\
                5 & & Wasserstein距离 & $d\ <\ 1.5$ & 中 \\
                6 & & 能量距离 & $d\ <\ 1.5$ & 中 \\
                \\
                7 & \multirow{3}{*}{连续丢包数分布} & CDF检验 & 趋势基本一致 & 中 \\
                8 & & K-L散度 & $K-L散度\ <\ 0.1$ & 高 \\
                9 & & Wasserstein距离 & $d\ <\ 1.5$ & 高 \\
                10 & & 能量距离 & $\ d\ <1.5$ & 高 \\
                \\
                11 & \multirow{2}{*}{区间丢包数分布} & CDF检验 & 趋势基本一致 & 中 \\
                12 & & Wasserstein距离 & $\ d\ <1.5$ & 高 \\
                13 & & 能量距离 & $\ d\ <1.5$ & 高 \\
                \bottomrule
            \end{tabular*}
            \begin{tablenotes}
                \footnotesize
                \item[1] Welch's t检验与Mann–Whitney rank检验通过一种即可
                \item[2] 量化测试项为$2 \sim 6,\ 8 \sim 10,\ 12 \sim 13$
            \end{tablenotes}
        \end{threeparttable}
    \end{table}
}
