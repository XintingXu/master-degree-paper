\section{概述}
\label{chap:zigzag:overview}

根据时间隐通道的构建指标,时间隐通道应满足抗检测能力、鲁棒性、传输性能、构建代价及保密性方面的要求。VoLTE中基于主动丢包的时间隐通道,能够满足抗检测能力的约束,构建方法需要满足其余指标。因此该构建方法中,通过结合Zigzag映射矩阵提高鲁棒性及保密性。

在鲁棒性方面,通过添加校验码字并结合Zigzag映射矩阵,将噪声影响离散化并通过校验验证实现去噪。借助CRC校验的确定性和随机性,调制过程中建立数据码字与校验码字的匹配关系,解调过程中通过重新验证校验关系区分噪声。

%该方法的创新点
该方法的创新点如下:
\begin{itemize}
	\item 提出了基于Zigzag映射矩阵的时间隐通道构建方法,并且具有足够的抗检测能力及鲁棒性;
	\item 基于Zigzag映射矩阵,建立了符号与码字之间的非线性映射;
	\item 引入了RTP中的随机字段,加强传输过程的随机化,增强隐蔽消息的保密性。
\end{itemize}

经实验验证,该时间隐通道在满足抗检测能力要求的前提下,保证了一定的传输能力,满足了时间隐通道构建指标的要求。