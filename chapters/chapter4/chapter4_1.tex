\section{概述}
\label{chap:zigzag:overview}

根据时间隐通道的构建指标,时间隐通道应满足抗检测能力、鲁棒性、传输性能、构建代价及保密性方面的要求。VoLTE中基于主动丢包的时间隐通道,在主动丢包比例较低时满足抗检测能力约束,需要在构建方法中满足其余指标。因此在该构建方法中,隐蔽消息被切分传输,通过结合Zigzag映射矩阵提高鲁棒性及保密性。

在鲁棒性方面,通过添加校验码字并结合Zigzag映射矩阵,将噪声影响离散化并通过校验进行去噪。借助CRC校验的确定性和随机性,生成数据码字与校验码字的配对关系,解调过程中重新验证校验关系即可有效区分噪声。

%该方法的创新点
该方法的创新点如下:
\begin{itemize}
	\item 提出了VoLTE中基于主动丢包的时间隐通道,并且具有足够的抗检测能力及鲁棒性;
	\item 基于Zigzag映射矩阵,建立了符号与码字之间的非线性映射;
	\item 引入了RTP中的随机字段,增加传输过程的随机化,增强隐蔽消息的保密性。
\end{itemize}

经过实验验证,该时间隐通道在满足抗检测能力的前提下,保证了一定的传输能力,满足了时间隐通道构建指标的要求。