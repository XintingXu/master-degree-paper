\section{概述}
\label{chap:zigzag:overview}

根据时间隐通道的评估指标,包括抗检测性、鲁棒性、传输性能、构建代价及保密性方面的要求,在VoLTE场景下通过主动丢包的方式构建时间隐通道,需要隐匿在网络噪声中的同时,与噪声的分布存在可识别的区分度。因此,在该构造方法中,采取分片的方式,将宿主信道的数据包流切分为等长的传输分片,每一个分片对应一个传输符号,通过控制分片长度满足不同场景下的隐蔽性要求。

在保证鲁棒性方面,结合了Zigzag矩阵的映射关系,并在码字中间添加校验码字,在一定程度上保证鲁棒性水平。基于Zigzag映射矩阵,将丢包中连续的突发丢包,转换为离散的待校验码字,将网络噪声转换为随机噪声,统一了解调阶段的去噪方法。借助CRC校验的确定性和随机性,在每个码字后追加一个基于CRC的校验码字,解调过程中,即可有效区分噪声及码字。

通过实验验证,通过调整传输参数,该时间隐通道能够在满足抗检测性的基础上,保持一定的传输能力,满足时间隐通道的应用场景要求。