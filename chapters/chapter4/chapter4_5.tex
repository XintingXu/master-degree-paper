\section{本章小结}
\label{chap:zigzag:summary}

本章主要介绍的是基于Zigzag映射矩阵的时间隐通道构建方法,并对该方法所构建隐通道的抗检测能力、鲁棒性、传输性能及构建代价进行了评估,实验结果表明,该方法可以在1.5\%误码率水平下实现0.88bps的隐蔽消息传输。通过各方面的抗检测性检验,该方法能够通过主流的检测方法,与网络噪声的特征十分相近。

该方法建立在VoLTE视频流中,模拟网络噪声中的随机丢包事件,将要传输的隐蔽消息调制到丢包序号中。借助RTP协议中的数据包序号字段,及序号唯一性特征,发送方调制后的隐蔽信道可以被接收方感知,从而确保传输过程的有效性。

由于网络噪声中的丢包事件,超过了调制过程中主动丢弃的数据包,该时间隐通道具有较低的信噪比,需要结合鲁棒性方法保证传输可靠性。本方法结合基于Zigzag的映射矩阵及基于CRC的码字校验,在不同方式上提高鲁棒性。通过基于Zigzag的映射矩阵,添加了码字与符号之间的随机对应关系,分离噪声中的丢包事件,提高码字校验的效率。借助基于CRC的码字校验,在码字中间添加校验信息,接收方通过重新检查校验规则,减弱网络噪声的干扰。

时间隐通道的保密性是非常重要的指标,即使监听方获取了时间隐通道的基本方法,也无法恢复隐蔽消息的内容。在该时间隐通道构建方法中,除了引入RTP自身的随机字段,同时添加了双方共享的秘密信息。通过结合秘密信息及随机字段,完成映射矩阵的初始化及CRC校验数据加盐,并且支持分组随机化,进一步提升保密性。监听方破解数据时,首先需要遍历映射矩阵的可行解,当每组映射矩阵不相同时,可行解空间呈指数增长。另一方面,在未获取加盐信息的情况下,网络噪声会严重干扰监听方对CRC重新校验,进一步提高了保密性。

因此,该时间隐通道构建方法,满足了时间隐通道的基本指标,并且在传输能力方面达到基本水平。