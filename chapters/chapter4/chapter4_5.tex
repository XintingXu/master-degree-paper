\section{本章小结}
\label{chap:zigzag:summary}

本章介绍了基于Zigzag映射矩阵的时间隐通道构建方法,并对该方法的抗检测能力、鲁棒性、传输性能及构建代价进行了评估。实验结果表明,该方法在{1.5\ \%}的误码率水平下,实现了{0.88\ bps}的隐蔽消息传输。通过抗检测能力测试,该方法能够通过严格的测试,并且与网络噪声具有相似的特征。

由于网络噪声导致的丢包超过了时间隐通道的丢包,该方法的信噪比较低,需要结合鲁棒性方法保证传输可靠性。本方法结合基于Zigzag的映射矩阵以及基于CRC的码字校验,在不同形式上提高鲁棒性。通过基于Zigzag的映射矩阵,分散噪声中的丢包事件,提高码字校验效率。借助基于CRC的码字校验,接收方通过重新核对校验信息,降低噪声干扰。

时间隐通道的保密性是重要指标,即使监听者获取了时间隐通道的基本方法,隐通道也要保证消息安全。该方法中,引入了RTP中的随机字段以及双方的共享信息。通过结合共享信息及随机字段,实现映射矩阵初始化及CRC校验数据加盐。监听者破解数据时,缺乏关键信息,计算复杂度升高。另一方面,网络噪声同样会严重干扰监听者,进一步提高了保密性。

因此,该时间隐通道构建方法,满足了时间隐通道的基本指标,并且在传输性能方面达到了预期水平。