
%% 如果只有论文成果,采用默认模式
%\begin{publications}{99}
   %\pubitem{二}{一}{Yu-an Tan, Xinting Xu, Chen Liang, Xiaosong Zhang, Quanxin Zhang and Yuanzhang Li}{An end-to-end covert channel via packet dropout for mobile networks}{J}{International Journal of Distributed Sensor Networks, 2018, 14(5), 1-14.(SCI期刊)}
   %% 盲审状态下显示:第二作者,导师第一作者+ An end-to-end covert channel via packet dropout formobile networks + International Journal of Distributed Sensor Networks, 2018, 14(5),1-14.(SCI期刊).
   %% 正常状态下显示:第二作者,导师第一作者. Yu-an Tan, Xinting Xu, Chen Liang, Xiaosong Zhang,Quanxin Zhang and Yuanzhang Li. An end-to-end covert channel via packet dropoutfor mobile networks[J]. International Journal of Distributed Sensor Networks, 2018,14(5), 1-14.(SCI期刊).
%\end{publications}
%% 默认论文成果结束

%% pubitem参数类型
%% {学生作者顺序}{老师作者顺序}{作者列表}{论文题目}{论文类型}{期刊或会议信息}
%% end

%% pubitemOthers参数类型
%% {学生作者顺序}{作者列表}{论文题目}{论文类型}{期刊或会议信息}
%% end

%% patentitem参数类型
%% {学生发明人顺序}{导师发明人顺序}{专利申请人}{专利名称}{专利申请号}{专利公开号}{专利公开时间}
%% end

%% projectitem参数类型
%% {项目类型}{项目名称}{起止时间}
%% end

%% XuXinting 添加的学术成果模式,包含三个部分:发表论文、申请专利和参与项目
%% 章标题只在researchsummary部分出现,每部分添加小标题

%% 发表论文列表
\begin{researchsummary}{99}
    \pubitem{二}{一}{Yu-an Tan, Xinting Xu, Chen Liang, Xiaosong Zhang, Quanxin Zhang and Yuanzhang Li}{An end-to-end covert channel via packet dropout for mobile networks}{J}{International Journal of Distributed Sensor Networks, 2018, 14(5), 1-14. (SCI期刊)}

    \pubitemOthers{三}{Yuanzhang Li, Xiaosong Zhang, Xinting Xu, Yu-an Tan}{A Robust Packet-Dropout Covert Channel over Wireless Networks}{J}{IEEE Wireless Communications, 2020, 27(3), 60-65. (SCI期刊)}

\end{researchsummary}

%% 申请专利列表,不需要可删
\begin{researchpatents}{99}
    \patentitem{二}{一}{北京理工大学}{一种基于主动丢包的时间隐通道鲁棒构建方法(发明专利)}{ZL201910648138.5}{(已授权)}
\end{researchpatents}

%% 参与项目列表,不需要可删
\begin{researchprojects}{99}
    \projectitem{国家自然科学基金联合基金重点项目}{面向移动互联网实时交互应用的时间隐通道(U1636213)}{2017.01-2020.12}
\end{researchprojects}
