%%==================================================
%% conclusion.tex for BIT Master Thesis
%% modified by yang yating
%% version: 0.1
%% last update: Dec 25th, 2016
%%==================================================


\begin{conclusion}

本文主要研究在VoLTE中通过主动丢包方式构建时间隐通道的方法,主要研究要点包括面向基于主动丢包的VoLTE时间隐通道检测方法、基于Zigzag映射矩阵的时间隐通道构建方法、基于多重校验的时间隐通道构建方法,以及基于Linphon的时间隐通道构建方法验证。其中,基于主动丢包的时间隐通道检测方法,提出了VoLTE中基于主动丢包时间隐通道的检测方法,提高本文中提出的时间隐通道构建方法具有足够的隐蔽性。基于Zigzag映射矩阵的时间隐通道构建方法,以及基于多重校验的时间隐通道构建方法,提出了两种基于主动丢包的时间隐通道构建方法,在抗检测性能力、鲁棒性、传输性能及构建代价方面均具有良好测试结果。基于Linphone的时间隐通道构建方法验证,借助Linphone中与VoLTE相同的SIP+RTP通话模式,验证通过主动丢包构建时间隐通道的可行性,并进行实际通话与传输测试。

本文主要研究内容及创新点总结如下:
\begin{enumerate}
    \item
    提出了一种主动丢包时间隐通道的检测方法,完善了VoLTE环境下时间隐通道检测方法,提高了时间隐通道构建水平。当前的时间隐通道检测方法,多针对IPD分布特征,基于主动丢包的时间隐通道对IPD分布的影响有限,常规检测方法准确率较低。因此,本方法中除了统计IPD分布外,同时也考虑了突发丢包长度及区间丢包数。为比较这三种分布统计之间的差异,除了常用的CDF检测、K-S检验及K-L散度检验,本方法中同时参考Welch's t检验、Wasserstein距离以及能量距离的测试结果,并约定当通过了所有检测子项,认定检测样本与参照样本一致,不存在时间隐通道特征。
    
    通过模拟时间隐通道进行检测测试,该时间隐通道检测方法达到了设计要求。在VoLTE通话的Excellent场景中,当主动丢包率不低于0.4\%时,该方法具有100\%检出能力;在VoLTE通话的Good场景中,当主动丢包率不低于1\%时,改方法具有100\%检出能力。对于基于IPD的检测方法来说,当主动丢包率低于2\%时,即可通过常用的K-S检验及K-L散度检验,证明该检测方法在检测能力上有了极大提升。
    
    \item
    提出了基于Zigzag映射矩阵的时间隐通道构建方法,同时考虑了抗检测能力、鲁棒性、传输性能及保密性方面的设计要求。在该方法中,主要的传输为码字长度,通过调整码字长度在各指标方面实现均衡。该方法的调制过程的主要流程包括三个部分,分别为消息分组、生成校验码字以及码字-符号映射;解调过程中对应为消息重组、有效码字鉴别以及符号-码字逆映射。其中,消息分组过程按照设定的码字长度,将隐蔽消息切分为等长消息块,解调过程按照顺序进行重组。
    
    为保证时间隐通道的鲁棒性,在进行消息块切分的基础上,计算每一个消息块对应的CRC校验值,作为校验码字插入到码字集合中。解调过程中,借助CRC的确定性进行码字与噪声鉴别,在一定程度上去除噪声干扰。Zigzag矩阵作用于符号与码字之间的转换,通过消除符号与码字间的线性关系,提高噪声干扰时的鲁棒性。通过对映射矩阵的起始值进行随机化,有效提高时间隐通道的保密性。通过实验测试,该方法具有良好的抗检测能力,传输性能可达到0.88\ bps,误码率在1.5\ \%左右,同时具有极低的构建代价。
    
    \item
    提出了基于多重校验的时间隐通道构建方法,重点提升时间隐通道的鲁棒性,降低传输误码率。基于主动丢包的时间隐通道虽然具有良好的隐蔽性,但信号与噪声相似度较高,具有极低的信噪比,单一的数据校验与纠错无法满足实际需求。本方法的校验模式主要包含三部分,分别为基于HASH的码字间校验方法、基于CRC的码字自校验方法以及基于异或的映射矩阵校验方法。其中,基于HASH的码字间校验方法建立了码字间的关联关系,利用后接收的码字验证已接收的码字。基于CRC的码字自校验,降低了编码密度,有效过滤噪声。基于异或的映射矩阵校验,在码字-符号转换过程添加额外的校验符号,在符合层面进行预先校验。
    
    在调制与解调过程中,逐层处理校验信息,通过每一重校验的模式减弱噪声强度,提高时间隐通道鲁棒性。此外,在计算校验信息时,引入随机信息及自定义盐值,提高时间隐通道的随机性与保密性。经过实验测试,该方法在保持传输速率为0.73\ bps时,误码率不高于0.08\ \%,在鲁棒性方面得到显著提升。同时在构建代价方面,通过客观指标评估视频质量损失,该方法产生的影响远小于网络噪声的影响。

    \item
    在Linphone平台中,验证了基于主动丢包的时间隐通道。借助Linphone与VoLTE在传输模式上的相似性,通过实现及测试证明了基于主动丢包的时间隐通道具有可行性。通过在Linphone中添加隐通道控制接口、隐通道传输接口及隐通道执行组件,提供了UI层时间隐通道控制命令,实现隐蔽消息通过主动丢包序号进行传送的功能。通过不同网络环境下的测试,基于主动丢包的时间隐通道符合SIP+RTP的VoIP传输模式,并且对应用及通话质量具有极低影响。
\end{enumerate}

通过一系列实验证明,基于主动丢包的时间隐通道构建方法,在抗检测能力方面能够通过严苛的统计检验,在实际通话场景中测试表明具有良好的可用性。通过结合本文中提出的时间隐通道构建方法,在鲁棒性方面具有极大的提升空间,并且在传输性能方面达到常见水平。在保密性方面,结合RTP传输过程中的随机字段,实现处理过程中的加盐及随机化,达到增强了传输过程的随机性。对于时间隐通道来说,在保证隐蔽性的前提下具有较低的主动丢包率,因此对用户体验及通话质量的影响有限,低于网络噪声产生的波动,因此在构建代价方面影响极小。

当前,基于主动丢包的时间隐通道已经具备良好的理论基础,并且通过了基本原理的测试。但实际应用的网络环境是复杂且动态变化的,时间隐通道需要一定的适应能力,从而充分利用信道资源,在保证数据传输效率及完整性的基础上提高性能。因此,该方法在以下几个方面仍然具有研究意义:
\begin{enumerate}
    \item
    时间隐通道与网络环境自适应,动态调整传输传输参数。经过实验测试表明,不同网络环境的特征存在差异,网络噪声水平也不尽相同。隐通道的构建目标,是在保持隐蔽性的前提下,以较高的质量传输更多的数据。因此,在隐通道传输开始前进行网络质量评估,同时在通话过程中对网络状况进行动态评估,从而实现传输参数与网络环境相适应,将信道资源最优化利用。通过该方式进行改进,能够挖掘传输潜力,在一定程度上提升隐通道指标。
    
    \item
    传输过程增加会话协商及传输确认,确保隐蔽消息完整性。在VoLTE通话中,时间隐通道具备双向传输能力,隐蔽消息接收方具备传输状态反馈的能力。在可靠性传输中,通常采用数据校验及传输确认的方式,确保接收方得到的消息完整无误。研究在时间隐通道中添加传输控制,在当前低误码率的基础上,将可靠度提升到100\ \%。但是,在保证完全可靠的情况下,隐通道的传输性能将受到显著影响,从而导致传输时间过长,存在一定的局限性。
    
    \item
    VoLTE场景下部署及测试,时间隐通道应用化。虽然Linphone与VoLTE在传输模式上类似,但在网络环境及传输稳定性方面弱于VoLTE,尤其是视频通话的稳定性及通话延迟。随着5G商用规模的扩大,基于5G的音视频通话也会采用类似VoLTE的模式,因此具有较长的生命周期。另一方面,即使音视频通话采用SRTP(Secure Real-time Transport Protocol)或DTLS(Datagram Transport Layer Security)进行数据加密,基于主动丢包的时间隐通道依然可行。因此,VoLTE中时间隐通道应用化,在当前及未来的网络场景中均具有应用价值。

\end{enumerate}

\end{conclusion}