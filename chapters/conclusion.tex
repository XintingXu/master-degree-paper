\begin{conclusion}

本文主要研究在VoLTE中通过主动丢包方式构建时间隐通道,主要研究要点包括VoLTE主动丢包时间隐通道的检测方法、基于Zigzag映射矩阵的时间隐通道构建方法、基于多重校验纠错的时间隐通道构建方法,以及基于Linphone的时间隐通道原型系统。其中,VoLTE主动丢包时间隐通道的检测方法,提出了检测基于主动丢包时间隐通道的方法,有助于提升时间隐通道的构建水平。基于Zigzag映射矩阵的时间隐通道构建方法,以及基于多重校验纠错的时间隐通道构建方法,提出了两种基于主动丢包的时间隐通道构建方法,在抗检测性能力、鲁棒性、传输性能及构建代价方面均具有良好测试结果。基于Linphone的时间隐通道原型系统,借助Linphone与VoLTE相同的SIP+RTP通话模式,验证通过主动丢包构建时间隐通道的可行性,并进行实际传输测试验证其有效性。

本文主要研究内容及创新点总结如下:
\begin{enumerate}
    \item
    提出了一种VoLTE主动丢包时间隐通道的检测方法,完善了VoLTE环境中的时间隐通道检测方法,有助于提高时间隐通道构建水平。当前的时间隐通道检测方法,多针对IPD分布特征。然而,基于主动丢包的时间隐通道对IPD分布影响有限,常规检测方法准确率较低。因此,本方法除了统计IPD分布外,同时考虑了连续丢包数分布及区间丢包数分布。为有效检测分布统计之间的差异,除了常用的CDF检测、K-S检验及K-L散度检验,本方法中同时参考Welch's t检验、Mann–Whitney rank检验、Wasserstein距离以及能量距离的测试结果。并约定只有通过了所有的检测子项,才认定不存在时间隐通道。
    
    基于模拟的时间隐通道进行检测能力测试,该时间隐通道的测试效果达到了设计要求。在VoLTE通话的Excellent场景中,当主动丢包率不低于{0.4\ \%}时,该方法具有{100\ \%}检出能力;在VoLTE通话的Good场景中,当主动丢包率不低于{1\ \%}时,该方法具有{100\ \%}检出能力。作为对比,仅采用基于IPD的检测方法时,主动丢包率低于{2\ \%}即可通过常用的K-S检验及K-L散度检验,证明该检测方法在检测能力方面有了较大提升。
    
    \item
    提出了基于Zigzag映射矩阵的时间隐通道构建方法,同时满足了抗检测能力、鲁棒性、传输性能及保密性方面的设计指标。该方法中,主要的参数为码字长度$L_{Codeword}$,通过调整码字长度在各指标方面实现均衡。该方法的调制过程主要包括三个流程,分别为消息分组、校验码字生成以及码字-符号映射;解调过程中对应为消息重组、有效码字鉴别以及符号-码字逆映射。其中,消息分组过程按照设定的码字长度,将隐蔽消息切分为等长消息块,解调过程按照顺序进行重组。
    
    为保证时间隐通道的鲁棒性,在消息块的基础上,计算每一个消息块对应的CRC校验值,作为校验码字插入到码字集合中。解调过程中,借助CRC的确定性鉴别码字与噪声,在一定程度上去除噪声干扰。Zigzag矩阵作用于符号与码字之间的转换,通过消除符号与码字间的线性关系,提高噪声干扰时的鲁棒性。通过映射矩阵随机初始化,有效提高时间隐通道的保密性。经过实验测试,该方法具有良好的抗检测能力,并且传输性能达到{0.88\ bps}时误码率在{1.5\ \%}左右,同时具有较小的构建代价。
    
    \item
    提出了基于多重校验纠错的时间隐通道构建方法,重点提升时间隐通道的鲁棒性,降低传输误码率。基于主动丢包的时间隐通道虽然具有良好的隐蔽性,但信号与噪声相似度较高,具有较低的信噪比,单一的数据校验与纠错无法满足实际需求。本方法的校验模式主要包含三部分,分别为基于HASH的码字间校验方法、基于CRC的码字自校验方法以及基于异或的映射矩阵校验方法。其中,基于HASH的码字间校验方法建立了码字间的关联关系,利用后接收的码字验证已接收的码字。基于CRC的码字自校验,降低了编码密度,有效过滤噪声。基于异或的映射矩阵校验,通过码字-符号转换过程的校验符号,在符号层面进行预先校验。
    
    解调过程中,逐层检验校验信息,通过每一重校验进行纠错,提高时间隐通道的鲁棒性。此外,计算校验信息时,引入随机信息及自定义盐值,提高时间隐通道的随机性与保密性。经过实验测试,该方法在传输速率达到{0.49\ bps}时,误码率不高于{0.08\ \%},鲁棒性得到提升。同时在构建代价方面,通过视频质量的客观评估指标评测,该方法产生的影响小于网络噪声的影响。

    \item
    Linphone平台中,搭建了基于主动丢包的时间隐通道原型系统。通过在Linphone中添加隐通道控制接口、隐通道传输接口及隐通道执行组件,提供了UI层时间隐通道控制命令,实现隐蔽消息通过主动丢包序号传送。借助Linphone与VoLTE在传输模式上的相似性,测试证明了基于主动丢包的时间隐通道具有可行性。根据不同网络环境下的测试结果,基于主动丢包的时间隐通道符合SIP+RTP的传输模式,并且对应用及通话质量具有较小影响。
\end{enumerate}

一系列实验证明,基于主动丢包的时间隐通道构建方法,在抗检测能力方面能够通过严苛的统计检验,在实际测试中具有良好的可用性。结合本文中提出的时间隐通道构建方法,能够有效提升鲁棒性,并且保持传输性能在时间隐通道的基本水平。在保密性方面,结合RTP包头中的随机字段,实现处理过程中的加盐及随机化,增强了传输过程的随机性。对于基于主动丢包的时间隐通道来说,在保证隐蔽性的前提下具有较低的主动丢包率。因此,本文中时间隐通道对用户体验及通话质量的影响有限,并且低于网络噪声产生的波动,在构建代价方面影响较小。

当前,基于主动丢包的时间隐通道已经具备良好的理论基础,并且通过了基本原理的测试。但实际的应用环境复杂且动态变化,时间隐通道需要一定的环境适应能力,才能充分利用信道资源,在保证数据完整性的基础上提高性能。因此,在以下几个方向仍然具有研究意义:
\begin{enumerate}
    \item
    时间隐通道与网络环境自适应,实现传输参数动态调整。经实验测试表明,不同网络环境的特征存在差异,网络噪声水平也不尽相同。隐通道的目标,是在保持隐蔽性的前提下,以较高的质量传输更多的数据。因此,在传输过程中进行网络质量评估,调整传输参数使其与网络环境相适应,能够挖掘传输潜力,在一定程度上提升隐通道指标。
    
    \item
    会话协商及传输确认,确保隐蔽消息完整性。VoLTE通话场景中,时间隐通道具备双向传输能力,隐蔽消息接收方能够反馈传输状态。在可靠性传输中,通常采用数据校验及传输确认的方式,确保接收方得到的消息完整无误。研究在时间隐通道中添加传输控制,在低误码率的基础上,将可靠度提升到{100\ \%}是有意义的研究方向。但要实现完全可靠,隐通道的传输性能将受到影响,存在一定的局限性。
    
    \item
    VoLTE场景中部署及测试,实现时间隐通道应用化。虽然VoIP与VoLTE传输模式类似,但在网络环境及传输稳定性方面弱于VoLTE,尤其是视频通话的稳定性及通话延迟。随着5G商用规模的扩大,基于5G的音视频通话也会采用类似的模式,因此该类时间隐通道具有较长的生命周期。另一方面,即使音视频通话采用SRTP(Secure Real-time Transport Protocol)或DTLS(Datagram Transport Layer Security)进行数据加密,基于主动丢包的时间隐通道依然可行。因此,VoLTE中时间隐通道应用化,在当前及未来的网络场景中均具有应用价值。

\end{enumerate}

\end{conclusion}