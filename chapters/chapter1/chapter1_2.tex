\section{国内外研究现状及难点}
\label{sec:intro:background}

本节说了什么

本节包含的内容

\subsection{基于移动互联网的时间隐通道}
\label{sec:intro:background:ctc}

在移动互联网环境下,构建时间隐通道,应当遵守怎样的约束

常用的构建方法

各种方法的总结,以及为何不能应用在VoLTE环境中

\subsection{时间隐通道的鲁棒性策略}
\label{sec:intro:background:robustness}

首先说明,时间隐通道因为自身的特性,在传输可靠性上,不如Overt Traffic

现有的时间隐通道方案,在保证鲁棒性反面,采用了怎样的手段,比如添加纠错信息、重传、特殊编码等

这些方法,对应用场景的限制及不足

\subsection{时间隐通道的检测方法}
\label{sec:intro:background:detect}

检测时间隐通道,通常从那几个要素方面进行考虑

检测判断,主要根据哪些因素来考量,包括分布、一致性

在该场景中,主要通过丢包的方法构建时间隐通道,所有,在现有的常用检测方法的基础上,还要对丢包的特征进行分析

%固定相只有物理交联结构的聚氨酯称为热塑性SMPU,而有化学交联结构称为热固性SMPU。热塑性和热固性形状记忆聚氨酯的形状记忆原理示意图如图\ref{fig:diagram}所示

%\begin{figure}
% \centering
% \includegraphics[width=0.75\textwidth]{chapters/chapter1/figures/figure1}
% \caption{热塑性形状记忆聚氨酯的形状记忆机理示意图}\label{fig:diagram}
%\end{figure}

%\begin{table}
%  \centering
%  \caption{水系聚氨酯分类} \label{tab:category}
%  \begin{tabular*}{0.9\textwidth}{@{\extracolsep{\fill}}cccc}
%  \toprule
%    类别			&水溶型		&胶体分散型		&乳液型 \\
%  \midrule
%    状态			&溶解$\sim$胶束	&分散		&白浊 \\
%    外观			&水溶型		&胶体分散型		&乳液型 \\
%    粒径$/\mu m$	&$<0.001$		&$0.001-0.1$		&$>0.1$ \\
%    重均分子量	&$1000\sim 10000$	&数千$\sim 20万$ &$>5000$ \\
%  \bottomrule
%  \end{tabular*}
%\end{table}
